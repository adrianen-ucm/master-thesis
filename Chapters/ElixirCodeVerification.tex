\chapter{Elixir Code Verification}
\label{cap:elixirCodeVerification}

\chapterquote{Do not fear mistakes - there are none}{Miles Davis}

This chapter shows a system to verify code written in a subset of the Elixir 
programming language.  First, we define a formal language to model a subset of
sequential Elixir code, named L2, which also allows introducing ghost
annotations intended to help the verifier to prove the verification conditions.
Then we show how to verify it by means of the \gls{ir} defined in Chapter
\ref{cap:intermediateRepresentation}.

We also show an overview of our implementation, written in Elixir itself,
together with some of its implementation details.

\section{The L2 verification language}

In this section, we define the syntax of the L2 verification language, together
with some procedures and definitions with the aim to verify sequential Elixir
code.

\subsection{Syntax}

Let us define the set $\Exp{2}$ of sequential Elixir expressions given by the 
following grammar:

\[
\begin{array}{rcll}
\Exp{2} \ni E & ::= & e & \{ \textrm{L1 expression} \} \\
& | & P = E & \{ \textrm{pattern matching} \} \\
& | & \emptyE & \{ \textrm{empty sequence} \} \\
& | & E_1; E_2 & \{ \textrm{sequence} \} \\
& | & \caseE~E~\doW & \{ \textrm{case distinction} \} \\
& & \quad P_1~\whenE~f_1 \rightarrow E_1 \\
& & \quad \vdots\\
& & \quad P_n~\whenE~f_n \rightarrow E_n \\
& & \Wend \\
& | & \ghostE~\doW~S~\Wend & \{ \textrm{L1 ghost statement} \} \\
\end{array}
\]

Here $P$ denotes a pattern from a set $\Pat$ of patterns, defined by the 
following grammar:

\[
\Pat \ni P ::= c \mid x \mid [\:] \mid [P_1 \mid P_2] \mid \{ P_1, \ldots, P_n \}
\]

Note that the guard expressions $f_1, \ldots, f_n$ in a $\caseE$ correspond to
L1 expressions, due to the constraints that Elixir's syntax poses on the syntax
of guards.

This language models a subset of the Elixir programming language with L1
expressions as their direct counterparts in Elixir, sequences as Elixir blocks,
$\caseE$ expressions, and a simplified version of its pattern matching
capabilities.  It also allows adding verification statements such as $\assertE$
and $\assumeE$ within $\ghostE$ blocks.

\subsection{Translation into L1}

In the following, given a set $A$, we use the notation $[A]$ to denote the set
of sequences of elements in $A$. If $x_1, \ldots, x_n \in A$ we use the notation
$[x_1, \ldots, x_n]$ to denote such a sequence. We also use a list comprehension
notation that is similar to the one in Haskell language. For example, $[(i, j)
\mid i \leftarrow [1, 2], j \leftarrow [3, 4, 5]]$ denotes the list $[(1, 3),
(1, 4), (1, 5), (2, 3), (2, 4), (2, 5)]$.

Let us define a function: $\trEXP{\_} : \Exp{2} \rightarrow [\Stm \times
\Exp{1}]$ that, given an expression $E$ in the source language, generates a
sequence of pairs $(S, e)$ where $S$ is the L1 statement that models the
semantics of $E$, and $e$ is a L1 expression that represents the result to which
$E$ is evaluated. Each pair returned by $\trEXP{\_}$ corresponds to an execution
path produced by control flow constructs such as $\caseE$.

We need an auxiliary function $\trMatch{\_}{\_} : \Exp{1} \times \Pat
\rightarrow \Exp{1}$ that, given an L1 expression $e$ and a pattern $P$, returns
another L1 expression that is a $\mathit{boolean}$ term and is evaluated to
$\true$ if and only if $e$ matches $P$. Its definition is as follows:

\[
\begin{array}{l}
\trMatch{e}{c} = e~\texttt{===}~{c}\\
\trMatch{e}{[\:]} = e~\texttt{===}~{[\:]} \\
\trMatch{e}{x} = \mathit{true}\\
\trMatch{e}{\{P_1,\ldots,P_n\}}\\
\qquad = \textit{is-tuple}(e)~\mathbf{and}~\textit{tuple-size}(e)~\texttt{===}~n~\mathbf{and}~(\\
\qquad\quad\quad\trMatch{\textit{elem}(e, 1)}{P_1}~\mathbf{and}~\dots~\mathbf{and}~\trMatch{\textit{elem}(e, n)}{P_n}\\
\qquad\quad)\\
\trMatch{e}{[P_1\mid P_2]}\\
\qquad = \textit{is-nelist}(e)~\mathbf{and}~\trMatch{\textit{hd}(e)}{P_1}~\mathbf{and}~\trMatch{\textit{tl}(e)}{P_2}
\end{array}
\]

Also, $\mathit{vars}(P)$ is a function that returns the L1 variables that appear
in a pattern $P$.

The $\emptyE$ expression is translated into an empty list:

\[
\trEXP{\emptyE} = [(\skipE, [])]
\]

This would be properly modeled as the atom \verb|nil| in Elixir, as it is the
result of an empty \verb|block| but, as we have not modelled atoms for the
moment, we set it as the empty list.

L2 expressions that are contained within the syntax of L1 are translated as they
are, but we generate an assertion to check if the singleton tuple which contains
that expression is a tuple. This assertion might seem trivially true, but it is 
generated to check that the expression $e$ is well-formed (i.e. all the function
applications within it satisfy their corresponding preconditions). Otherwise,
ill formed expressions at the top level whose translation yields to verification
failures may be ignored during the translation (e.g. \verb|true + 2|):

\[
\trEXP{e} = [(\assertE~\textit{is\mbox{-}tuple}(\{e\}), e)]
\]

An alternative to generating this trivial assertion would be to extend the L1
syntax with a new statement that just checks the well-formedness of an L1
expression.

A $\ghostE$ expression is translated into an arbitrary L1 expression, for
example the empty list, and the provided L1 statement as its semantics:

\[
\trEXP{\ghostE~\doW~S~\Wend} = [(S, [])]
\]

Expressions of the form $P = E$ are translated into assertions that check
whether the result of evaluating E matches the pattern $P$ and then assume the
equality between $P$ and $E$, as shown in Figure \ref{fig:deftr}.

\begin{figure}
\[
\begin{array}{l}
\trEXP{P = E} = [(S_1;S_1',e_1), \dots, (S_n;S_n', e_n)] \\
\qquad 
\begin{array}{ll}
\textbf{where} & [(S_1,e_1),\dots,(S_n, e_n)] = \trEXP{E} \\
& \{y_1, \ldots, y_m\} = \mathit{vars}(P) \\
& \forall i \in \{1..n\}: S_i' = \left(
\begin{array}{l}
\assertE~\trMatch{e_i}{P};\\
\havocE~y_1;\\
\vdots\\
\havocE~y_m;\\
\assumeE~e_i ~\texttt{===}~ P\\
\end{array}
\right)
\end{array}
\end{array}
\]
\caption{Translation of the L2 pattern matching expression into L1}
\label{fig:pattr}
\end{figure}

In order to translate a sequence of expressions $E_1;E_2$ we have to append 
every statement generated from the translation of $E_2$ to every statement 
generated from the translation of $E_1$. We must deal carefully with $\ghostE$
expressions because, as its translation also returns an L1 expression, it must
be skipped in order to avoid altering the semantics of a block. That is why we 
have to distinguish cases depending on whether $E_2$ is a ghost expression or
not.

The following clause describes the translation of a sequence when the last
expression is a $\ghostE$ one, taking into account that \verb|;| is associative:

\[
\begin{array}{l}
\trEXP{E;\ghostE~\doW~S~\Wend} = [(S_i;S,e_i) \mid i \leftarrow [1..n]] \\
\qquad 
\begin{array}{ll}
\textbf{where} & [(S_1,e_1),\dots,(S_n, e_n)] = \trEXP{E} \\
\end{array}
\end{array}
\]

And the following shows the general case when the previous one does not apply
(i.e.  $E_2$ is not a sequence ending in a $\ghostE$):

\[
\begin{array}{l}
\trEXP{E_1;E_2} = [(S_i;S_j',e'_j) \mid i \leftarrow [1..n], j \leftarrow [1..m]] \\
\qquad 
\begin{array}{ll}
\textbf{where} & [(S_1,e_1),\dots,(S_n, e_n)] = \trEXP{E_1} \\
& [(S_1',e'_1),\dots,(S_m', e'_m)] = \trEXP{E_2} \\
\end{array}
\end{array}
\]

The translation of $\caseE$ expressions, shown in Figure \ref{fig:casetr}, is
more complex. It can be described as, for each translation of $E$, for each
branch and for each translation of the resulting expression of that branch:

\begin{figure}
\[
\begin{array}{l}
\trEXP{\caseE~E~\doW~\overline{P_i~\whenE~f_i \rightarrow E_i}^n~\Wend}\\
\qquad = \left[ \begin{array}{l}
(S_j;\\
\havocE~y_{1};\\
\vdots\\
\havocE~y_{t};\\
\assertE~(\andE{e_{1,j}}{(\orE{(\textbf{not}~e_j~\texttt{===}~P_1)}{f_1})})\textbf{or}~\cdots\\
\quad\quad\textbf{or}~(\andE{e_{n,j}}{(\orE{(\textbf{not}~e_j~\texttt{===}~P_n)}{f_n})}); \\
\assumeE~(\textbf{not}~(e_{1,j}~\textbf{and}~(\orE{(\textbf{not}~e_j~\texttt{===}~P_1)}{f_1})))~\textbf{and}~\cdots\\
\quad\quad\textbf{and}~(\textbf{not}~(e_{i-1,j}~\textbf{and}~(\orE{(\textbf{not}~e_j~\texttt{===}~P_{i - 1})}{f_{i - 1}}))); \\
\assumeE~e_j~\texttt{===}~P_i; \\
\assumeE~e_{i,j}~\textbf{and}~f_{i};\\
S_{i,k}', e_{i,k}')
\end{array}  \middle\vert
\begin{array}{l}
j \leftarrow [1..m], \\
i \leftarrow [1..n], \\
k \leftarrow [1..s_i] \\
\end{array}
\right]\\
\begin{array}{ll}
\textbf{where} & [(S_1,e_1),\ldots,(S_m,e_m)] = \trEXP{E} \\
& \forall i \in \{1..n\}: [(S'_{i,1},e'_{i,1}),\ldots,(S'_{i,s_i},e'_{i,s_i})] = \trEXP{E_i} \\
& \forall i \in \{1..n\}, j \in \{1..m\}: e_{i,j} = \trMatch{e_j}{P_i} \\
& \{y_{1}, \ldots, y_{t}\} = \bigcup_{i = 1}^n\mathit{vars}(P_i) \\
\end{array}
\end{array}
\]
\caption{Translation of the L2 case expression into L1}
\label{fig:casetr}
\end{figure}

\begin{enumerate}
  \item Declare the variables involved in every pattern matching. This is because guards $f_i$ may refer to variables bound in its pattern.
  \item Check that at least one pattern and guard holds. The guards $f_i$ are checked under the assumption that their pattern variables have been bound, which is a disjunction because there is no implication connective in L1 expressions (e.g. $((\mathbf{not} e_j === P_1)~\textbf{or}~f_1)$ instead of $((e_j === P_1) \Rightarrow f_1)$).
  \item Assume that no previous pattern and guard holds, with the same considerations explained above, and that the current one does.
  \item Assume that the branch pattern does match.
  \item Include the generated L1 statements for the expression corresponding to that branch.
\end{enumerate}

This models appropriately the short-circuit semantics of the \verb|case|
construct because, given a value for the discriminant on a branch that is not 
going to be evaluated, its pattern and guards are assumed, which will make the 
proof state inconsistent, hence validating every assertion afterwards.

\subsection{Verifying L2 expressions}
\label{sec:verificationprocess}

Before dealing with function definitions, in this section we show how to verify 
top-level L2 expressions.

First, for the verification of an L2 expression, as Elixir allows rebinding
local variables, but this would be problematic in the generated L1 counterpart
since it could lead to name conflicts, we consider a function $ssa$ to transform
an L2 expression into \gls{ssa} form \citep{rosen1988global} as in the following
example:

$$\text{if}~E =  (x = 2;x = 3 + x; y = x * x)$$
$$\text{then}~\mathit{ssa}(E) = (x_1 = 2;x_2 = 3 + x_1;y_1 = x_2 * x_2)$$

Then, we can obtain a single L1 statement that considers all the generated
execution paths by wrapping them inside $\blockE$ expressions to be verified 
independently of each other:

\[
\begin{array}{l}
\mathit{verification}(E) = \blockE~S_i;\dots;\blockE~S_n\\
\qquad 
\begin{array}{ll}
\textbf{where} & [(S_1,e_1),\dots,(S_n, e_n)] = \trEXP{\mathit{ssa}(E)}
\end{array}
\end{array}
\]

Finally, the resulting expression from our \gls{ir} can be further translated
into our low level verification language L0, as shown in Section
\ref{sec:translationL0} with the $\trStm{\_}$ function, which has its semantics
defined in terms of the $\gls{smt}$ problem.

We also define a function $\mathit{elixir}(E)$ which turns an L2 expression $E$
into its counterpart with all its $\ghostE$ subexpressions removed from every 
sequence expression. If $E$ itself is one of them, then the result is $\emptyE$.

\subsection{Verifying user-defined functions}
\label{sec:verifyuserdefined}

We will also define formally a representation for a set of user-defined
functions with their different overloads, which correspond to Elixir functions
that a user can define in an Elixir \verb|module|.

A single clause of a function with arity $n$ is stated as

\[
\mathit{def} \equiv \left(\{ p \}  \quad (P_1, \dots, P_n)~B \quad \{ q \} \right)
\]

where $p \in \Exp{1}$ and $q \in \Exp{1}$ denote a specified precondition and a
postcondition, $P_1, \dots, P_n$ are the parameter patterns and $B \in \Exp{2}$
is its defined body. The pre/postcondition may refer to the variables of the
patterns $P_1, \dots, P_n$. 

Given a function named $f$ with arity $n$, we denote its clauses by 

$$\Defs{f/n} = (\mathit{def}_1,\dots,\mathit{def}_k)$$

where the order between clauses matters, since the clauses of the definition may
contain overlapping patterns.  For each function, we are going to translate its
clauses into a single $\caseE$ L2 expression that models them, as in Figure
\ref{fig:deftr}.

\begin{figure}
\[
\begin{array}{l}
\trDef{f/n} = \left[ \begin{array}{l}
\ghostE~\doW\\
\quad\havocE~\mathit{arg}_1;\\
\quad\vdots\\
\quad\havocE~\mathit{arg}_n\\
\Wend;\\
\caseE~\{\mathit{arg}_1,\dots,\mathit{arg}_n\}~\doW \\
\quad \{P_{1,1},\dots,P_{1,n}\}~\whenE~p_1 \rightarrow\\ 
\quad\quad \mathit{res} = B_1; \\
\quad\quad \ghostE~\doW\\ 
\quad\quad\quad\assumeE~res~\mathit{===}~f(\mathit{arg}_1,\dots,\mathit{arg}_n);\\
\quad\quad\quad\assertE~q_1\\
\quad\quad \Wend\\
\quad\vdots\\
\quad \{P_{k,1},\dots,P_{k,n}\}~\whenE~p_k \rightarrow\\ 
\quad\quad \mathit{res} = B_k; \\
\quad\quad \ghostE~\doW\\ 
\quad\quad\quad\assumeE~res~\mathit{===}~f(\mathit{arg}_1,\dots,\mathit{arg}_n);\\
\quad\quad\quad\assertE~q_k\\
\quad\quad \Wend\\
\quad \{\mathit{arg}_1,\dots,\mathit{arg}_n\} \rightarrow\\
\quad\quad\true\\ 
\Wend
\end{array}
\right]\\
\textbf{where } (\mathit{def}_1, \dots, \mathit{def}_k) = \Defs{f/n}\\
\phantom{\textbf{where }} \mathit{def}_i = \left(\{ p_i \}  \quad (P_{i,1}, \dots, P_{i,n})~B_i \quad \{ q_i \} \right) 
\end{array}
\]
\caption{Translation of a user-defined function into an L2 expression to verify}
\label{fig:deftr}
\end{figure}

We have used the fact that the rules that Elixir uses to select which function
clause takes effect are similar to those of the Elixir \verb|case| expression,
using the guards to assume the preconditions and its branch pattern matching to
allow using pattern matching also in the function parameters. The $\assumeE$
statement over the auxiliary variable \textit{res} allows the postcondition to
refer to the result of the function invocation as $f(\mathit{arg_1}, \dots,
\mathit{arg_2})$.

Also, a branch at the end that always succeeds is required because, as the 
function arguments are unknown at this point, it is not possible to know in 
advance if no branch will match. An alternative to this would be to assume that
some branch will match.

A set of function definitions can be verified by applying $\trDef{\_}$ to each
one of them and also applying the $\mathit{verification}$ function shown in
Section \ref{sec:verificationprocess}. Future work may require allowing the user
to unfold a function definition as stated in this section, with multiple clauses
and their bodies provided as L2 expressions, in the same way that $\unfoldE$
allowed to unfold a single function definition clause with its body provided as
an L1 expression in Section \ref{sec:translationL0}.

\subsection{Termination}

Regarding termination, in this thesis we do not verify whether a recursive
function definition terminates or not, which could be addressed by means of a
ranking function over its arguments. 

This ranking function could be supplied by the user, and may depend on the sizes
of the arguments. For this purpose, we have proposed a set of axioms based on
$\Term$ types in Section \ref{sec:termsize}, which can help in proving that the
destructuring of a term always leads to smaller subterms.

\section{Implementation}

This section shows our early implementation of the system and the current
\gls{api} for its usage. It also explains some of its implementation details 
briefly.

\subsection{Overview}

The package that we provide, named \verb|Verixir|, can be imported 
with the \verb|use| macro in order to add the capabilities of our system.

The preconditions can be specified with \verb|module| attributes as in the 
following example:

\begin{lstlisting}[language=elixir,numbers=none,frame=none]
defmodule Example do
  use Verixir 

  @verifier requires is_integer(x)
  defv dup(x) do
    x + x
  end

  @verifier ensures uses_dup(y) === 2 * y
  defv uses_dup(y) when is_integer(y) do
    ghost do 
      unfold dup(y)
    end

    dup(y)
  end
end
\end{lstlisting}

which, on the one hand, verifies the functions defined with the \verb|defv|
macro at compile-time and, on the other hand, generates their corresponding
definitions without verification code that will be finally compiled.

If we change the literal \verb|2| from the example by \verb|3|, we get the
following error:

\begin{Verbatim}[commandchars=\\\{\}]
      \textcolor{red}{[error] Verification: Assert failed uses_dup(y_1) === 3 * y_1}
\end{Verbatim}

\subsection{Translation}
For the translation process, we have implemented a function that given Elixir \gls{ast}
code that corresponds to an L2 program (i.e. its \gls{dsl}), yields a list of
pairs with the expression in L1 that represents its resulting value, and an L1
statement that models its meaning:

\begin{lstlisting}[language=elixir,numbers=none,frame=none]
@spec translate_l2_exp(L2Exp.ast) 
  :: [{L1Exp.ast, L1Stm.ast}]
\end{lstlisting}

As in the previous \gls{dsl} translations, its definition syntax matches closely
the formal version, like in this example for assignment with pattern matching:

\begin{lstlisting}[language=elixir,numbers=none,frame=none]
def translate_l2_exp({:=, _, [p, e]}) do
  for {t, sem} <- translate(e) do
    {
      t,
      quote do
        unquote(sem)
        assert unquote(translate_match(p, e))

        unquote_splicing(
          for var <- vars(p) do
            quote do
              havoc unquote(var)
            end
          end
        )

        assume unquote(t) === unquote(p)
      end
    }
  end
end
\end{lstlisting}

where we also have implemented an auxiliary function to obtain the variables of
a pattern as L1 variable expressions

\begin{lstlisting}[language=elixir,numbers=none,frame=none]
@spec vars(Pat.ast) :: MapSet.t(L1Exp.ast)
\end{lstlisting}

and a function to perform the pattern matching translation:

\begin{lstlisting}[language=elixir,numbers=none,frame=none]
@spec translate_match(Pat.ast, L2Exp.ast) :: L1Exp.ast
\end{lstlisting}

For example, this is the case for lists:

\begin{lstlisting}[language=elixir,numbers=none,frame=none]
def translate_match({:|, _, [p1, p2]}, e) do
  tr_1 = translate_match(p1, quote(do: hd(unquote(e))))
  tr_2 = translate_match(p2, quote(do: tl(unquote(e))))

  quote(
    do:
      is_list(unquote(e)) and unquote(e) !== [] and
        unquote(tr_1) and unquote(tr_2)
  )
end
\end{lstlisting}

\subsection{Verification}

As a helper function for verification, we have also implemented one to transform
an L2 program into \gls{ssa} form:

\begin{lstlisting}[language=elixir,numbers=none,frame=none]
@spec l2_ssa(L2Exp.ast) :: L2Exp.ast
\end{lstlisting}

We could not use the built-in \verb|Macro.traverse/4| for it, which would be as
usual when transforming an \gls{ast} by propagating a state, because the pattern
matching case was difficult to handle. It was easier in contrast by defining an
explicit recursive function:

\begin{lstlisting}[language=elixir,numbers=none,frame=none]
defp ssa_rec({:=, _, [p, e]}, state) do
  {e, state} = ssa_rec(e, state)
  state = new_version_for_vars(var_names(p), state)
  {p, state} = ssa_rec(p, state)

  {
    {:=, [], [p, e]},
    state
  }
end
\end{lstlisting}

By using first the function to transform L2 code into \gls{ssa} and then the L1
verification function from Section \ref{ir:details}, we can define a
verification function for L2 code as follows:

\begin{lstlisting}[language=elixir,numbers=none,frame=none]
@spec verify_l2(Env.t(), L2Exp.ast()) :: [term()]
def verify_l2(env, e) do
  for {_, sem} <- translate_l2_exp(l2_ssa(e)) do
    L1Stm.eval(
      env,
      quote do
        block do
          unquote(sem)
        end
      end
    )
  end
  |> List.flatten()
end
\end{lstlisting}

Note that each possible path of the translation must be verified in an
independent proof context, so one option is to wrap each one into a \verb|block|
statement. Another one could be to use a fresh \acrshort{smt} solver connection
for each path.

\subsection{User-defined functions verification}

In order to provide an \gls{api} for programmers to use this verification system
in their own Elixir \verb|module|s, we have defined a module that can be
imported by means of the \verb|use| built-in macro. This section deals with
Elixir concepts that have not been explained in this document.

First, we have defined a macro \verb|defv| to write functions involved in the
verification process. Those defined with this macro are collected during the
\verb|module| compilation, where preconditions and postconditions are specified
as \verb|module attributes|, replaced by regular function definitions with no
verification code, and then verified by translating them into $\caseE$
expressions as explained in Section \ref{sec:verifyuserdefined}.

The function to remove verification code has been implemented as follows:

\begin{lstlisting}[language=elixir,numbers=none,frame=none]
@spec remove_verification(L2Exp.ast()) :: L2Exp.ast()
def remove_verification(ast) do
  Macro.prewalk(ast, fn
    {:__block__, meta, es} ->
      {:__block__, meta,
       Enum.reject(es, fn
         {:ghost, _, _} -> true
         _ -> false
       end)}

    {:ghost, _, _} ->
      {:__block__, [], []}

    other ->
      other
  end)
end
\end{lstlisting}

\begin{lstlisting}[language=elixir,numbers=none,frame=none]
  
\end{lstlisting} 