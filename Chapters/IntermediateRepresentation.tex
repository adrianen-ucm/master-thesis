\chapter{The L1 Intermediate Representation}
\label{cap:intermediateRepresentation}

\chapterquote{This process of using tools you built yesterday to help build bigger tools today is called abstraction, and it is the most powerful force I know of in the universe}{Sandy Maguire}

\textcolor{red}{TODO}

\section{Syntax}

We denote by $\Sigma^{1} = \{ \texttt{===}, \texttt{<=}, \texttt{>=}, 
\texttt{+}, \texttt{-}, \dots \}$ the set of operators and functions allowed 
in L1. We assume that for every element $f \in \Sigma^{1}$ there is an 
uninterpreted function symbol in $\Sigma^{0}$, which will be denoted by 
$\widehat{f}$. If $f$ has arity $n$, its corresponding function symbol 
$\widehat{f}$ will have sort $\Term \times \stackrel{n}{\dots} 
\times \Term \rightarrow \Term$.

Let us define the syntax of L1 expressions and statements:

\[
\begin{array}{rcll}
\Exp{1} \ni e & ::= & c & \{ \textrm{literal} \}\\
& | & x & \{ \textrm{variable} \}\\
& | & \andE{e_1}{e_2} & \{ \textrm{conjunction} \}\\
& | & \orE{e_1}{e_2} & \{ \textrm{disjunction} \}\\
& | & [] & \{ \textrm{empty list} \}\\
& | & [e_1~|~e_2] & \{ \textrm{list constructor} \}\\
& | & \{e_1, \ldots, e_n\} & \{ \textrm{tuple} \}\\
& | & f(e_1, \ldots, e_n) & \{ \textrm{function or operator application} \}\\

\\
\Stm \ni S & ::= & \skipE\\
 & | & \blockE~S\\
 & | & \havocE~x\\
 & | & S_1;S_2\\
 & | & \assumeE~e\\
 & | & \assertE~e\\
 & | & \unfoldE~f(e_1, \dots, e_n)\\
\end{array}
\]

\section{Semantics}

\textcolor{red}{TODO}

\subsection{Built-in declarations}

\textcolor{red}{TODO}

Required SMT-LIB preamble for the translation.

During the translation of L1 expressions we require some 
defined sorts, constants and functions. In our implementation,
the SMT-LIB commands with that purpose are the following:

\begin{verbatim}
(declare-sort Term 0)
(declare-sort Type 0)
(declare-fun type (Term) Type)
(declare-fun term_size (Term) Int)
(declare-fun integer_val (Term) Int)
(declare-fun boolean_val (Term) Bool)
(declare-fun integer_lit (Int) Term)
(declare-fun boolean_lit (Bool) Term)
(declare-fun tuple_size (Term) Int)
(declare-fun elem (Term Int) Term)
(declare-fun nil () Term)
(declare-fun cons (Term Term) Term)
(declare-fun hd (Term) Term)
(declare-fun tl (Term) Term)
(declare-const int Type)
(declare-const bool Type)
(declare-const tuple Type)
(declare-const nonempty_list Type)
(assert (distinct int bool))
(assert (distinct int tuple))
(assert (distinct int nonempty_list))
(assert (distinct bool tuple))
(assert (distinct bool nonempty_list))
(assert (distinct tuple nonempty_list))
(define-fun is_integer ((x Term)) Bool (= (type x) int))
(define-fun is_boolean ((x Term)) Bool (= (type x) bool))
(define-fun is_tuple ((x Term)) Bool (= (type x) tuple))
(define-fun is_nonempty_list ((x Term)) Bool (= (type x) nonempty_list))
(define-fun is_list ((x Term)) Bool (or (= x nil) (= (type x) nonempty_list)))
\end{verbatim}

Also, tuple constructors for any size $n$ must be declared with sort 
$\Term \times \stackrel{n}{\dots} 
\times \Term \rightarrow \Term$, but 
in our implementation we declare each one the first time that it is required.

\subsection{Built-in specifications}

\textcolor{red}{TODO}

Built-in specifications and SMT-LIB code to emulate the 
Elixir semantics.

We shall also assume that for every function symbol $f \in \Sigma^{1}$ of 
arity $n$ there is an overloaded specification expressed in terms of L0 
formulae. Here the word \emph{overloaded} means that there could by many 
pre/post-condition pairs for each function. For example, equality can be 
specified as follows:

\[
\begin{array}{l}
\{ \isinteger{x} \wedge \isinteger{y} \}\\
x~\texttt{===}~y\\
\{ \booleanvalue{\widehat{===}(x, y)} \Leftrightarrow \integervalue{x} = \integervalue{y} \}\\
\\
\{ \isboolean{x} \wedge \isboolean{y} \}\\
x~\texttt{===}~y\\
\{ \booleanvalue{\widehat{===}(x, y)} \Leftrightarrow \booleanvalue{x} = \booleanvalue{y} \}\\
\\
\vdots
\\
\\
\{\true\}\\
x~\texttt{===}~y\\
\{ \isboolean{\widehat{===}(x, y)} \wedge \booleanvalue{\widehat{===}(x, y)} \Leftrightarrow (x = y) \}\\
\\
\end{array}
\]

Here $\widehat{===}$ is the uninterpreted symbol in $\Sigma^{0}$ corresponding 
to Elixir's strict equality operator $\texttt{===} \in \Sigma^{1}$. We write 
the former in prefix form in order to highlight the fact that it is an 
uninterpreted function symbol in the logic. On the contrary, the $=$, 
$\Leftrightarrow$, $\wedge$ in the specification above are actual connectives 
and operators of the underlying logic.

These specifications are built-in into the system. They are not provided by 
the user. If we apply an operator 
that has several pre/postcondition pairs, we check all the preconditions and 
only assume those postconditions corresponding to the preconditions that hold.

Therefore, each function $f \in \Sigma^{1}$ of arity $n$ may have several 
specifications $\sigma_1, \ldots, \sigma_m$. Each specification is a pair 
$(\varphi(x_1, \ldots, x_n), \psi(x_1, \ldots, x_n))$, where the $x_i$ 
variables denote the parameters of the function. We denote by $\Spec{f}$ 
the set of specifications of $f$.

In order to allow L1 expressions to model the semantics of Elixir, the 
corresponding uninterpreted functions must be declared in SMT-LIB
with sort $\Term \times \stackrel{n}{\dots} 
\times \Term \rightarrow \Term$. We provide 
some built-in specifications which are explained in this section.

For integer arithmethic, the specification of $\texttt{+}$ can be as

\[
\begin{array}{l}
\{ \isinteger{x} \wedge \isinteger{y} \}\\
x~\texttt{+}~y\\
\{ \isinteger{\widehat{+}(x, y)} \wedge 
  \integervalue{\widehat{+}(x, y)} = \integervalue{x} + \integervalue{y} \}
\end{array}
\]

and similar for $\texttt{-}$ and $\texttt{*}$. The unary 
version of $\texttt{-}$ can be specified as follows:

\[
\begin{array}{l}
\{ \isinteger{x} \}\\
\texttt{-}~x\\
\{ \isinteger{\widehat{-}(x)} \wedge 
  \integervalue{\widehat{-}(x)} = -\integervalue{x} \}
\end{array}
\]

It is similar to the Elixir boolean negation:

\[
\begin{array}{l}
\{ \isboolean{x} \}\\
\mathit{not}(x)\\
\{ \isboolean{\widehat{\mathit{not}}(x)} \wedge 
  \booleanvalue{\widehat{\mathit{not}}(x)} \Leftrightarrow \neg\booleanvalue{x} \}
\end{array}
\]

We have only provided the comparison for integer terms as

\[
\begin{array}{l}
\{ \isinteger{x} \wedge \isinteger{y} \}\\
x~\texttt{<}~y\\
\{ \isboolean{\widehat{<}(x, y)} \wedge 
  \booleanvalue{\widehat{<}(x, y)} \Leftrightarrow \integervalue{x} < \integervalue{y} \}
\end{array}
\]

and it is similar for $\texttt{>}$, $\texttt{<=}$ and $\texttt{>=}$. An improvement would
be to extend this for any term, including lists and tuples. Term equality 
can be specified as

\[
\begin{array}{l}
\{ \isinteger{x} \wedge \isinteger{y} \}\\
x~\texttt{===}~y\\
\{ \booleanvalue{\widehat{===}(x, y)} \Leftrightarrow \integervalue{x} = \integervalue{y} \}\\
\\
\{ \isboolean{x} \wedge \isboolean{y} \}\\
x~\texttt{===}~y\\
\{ \booleanvalue{\widehat{===}(x, y)} \Leftrightarrow \booleanvalue{x} = \booleanvalue{y} \}\\
\\
\{ \islist{x} \wedge \islist{y} \}\\
x~\texttt{===}~y\\
\{ \booleanvalue{\widehat{===}(x, y)} \Leftrightarrow 
  (x = \mathit{nil} \wedge y = \mathit{nil})
  \vee (\mathit{hd}(x) = \mathit{hd}(y) \land \mathit{tl}(x) = \mathit{tl}(y)) \}\\
\\
\{ \istuple{x} \wedge \istuple{y} \wedge \mathit{tuple\mbox{-}size}(x) = \mathit{tuple\mbox{-}size}(y) \}\\
x~\texttt{===}~y\\
\{ \booleanvalue{\widehat{===}(x, y)} \Leftrightarrow 
  (\forall i. i >= 0 \wedge i < \mathit{tuple\mbox{-}size}(x) \Rightarrow \mathit{elem}(x, i) = \mathit{elem}(y, i)) \}\\
\\
\{ \istuple{x} \wedge \istuple{y} \wedge \mathit{tuple\mbox{-}size}(x) \neq \mathit{tuple\mbox{-}size}(y) \}\\
x~\texttt{===}~y\\
\{ \neg\booleanvalue{\widehat{===}(x, y)} \}\\
\\
\{\true\}\\
x~\texttt{===}~y\\
\{ \isboolean{\widehat{===}(x, y)} \wedge \booleanvalue{\widehat{===}(x, y)} \Leftrightarrow (x = y) \}\\
\\
\end{array}
\]

and it is also similar for $\texttt{!==}$.

The $\mathit{tuple\mbox{-}size}$ and $\mathit{elem}$ functions can be specified directly 
in terms of the built-in declarations used during the translation:

\[
\begin{array}{l}
\{ \istuple{x} \}\\
\mathit{tuple\mbox{-}size}(x)\\
\{ \isinteger{\widehat{\mathit{tuple\mbox{-}size}}(x)} \wedge 
  \integervalue{\widehat{\mathit{tuple\mbox{-}size}}(x)} = \mathit{tuple\mbox{-}size}(x) \}
\end{array}
\]

\[
\begin{array}{l}
\{ \istuple{x} \wedge \isinteger{i} \wedge \integervalue{i} >= 0 \wedge \integervalue{i} < \mathit{tuple\mbox{-}size}(x) \}\\
\mathit{elem}(x, i)\\
\{ \widehat{\mathit{elem}}(x, i) = \mathit{elem}(x, \integervalue{i}) \}
\end{array}
\]

The same can be applied to the $\mathit{hd}$ function

\[
\begin{array}{l}
\{ \isnelist{x} \}\\
\mathit{hd}(x)\\
\{ \widehat{\mathit{hd}}(x) = \mathit{hd}(x) \}
\end{array}
\]

and it is similar for $\mathit{tl}$. Note that, in these last examples, 
the L1 function is not the same as the one mentioned in the postcondition,
which is a built-in L0 function although we have used the same name.

The functions to mention the term types can also be specified 
directly with the built-in declared L0 functions:

\[
\begin{array}{l}
\{ \true \}\\
\isinteger{x}\\
\{ \isboolean{\widehat{\isinteger{x}}} \wedge 
  \booleanvalue{\widehat{\isinteger{x}}} \Leftrightarrow \isinteger{x} \}
\end{array}
\]

and it is similar for the remaining types.

\subsection{Translation into L0}

\textcolor{red}{TODO}

We shall define two functions:

\[
\begin{array}{ll}
\trExp{\_}{\_} : & \Exp{0} \times \Exp{1} \rightarrow \Exp{0} \times \mathbb{T}\\
\trStm{\_} : & \Stm \rightarrow \Exp{0}
\end{array}
\]

Given an L1 expression $e$, the application $\trExp{\gamma}{e}$ returns a tuple 
$(\epsilon, t)$, in which $\epsilon$ is an L0 expression that models the
semantics of $e$, and $t$ is the term in the underlying logic that will 
be used to refer to the result of $e$. The $\gamma$ models those facts that are 
known by the time $e$ is evaluated. This is needed to handle the short circuit-based 
semantics of $\mathbf{and}$ and $\mathbf{or}$. We are going to omit this $\gamma$
parameter when it models no knowledge:

\[
\trExps{e} \equiv \trExp{\skipE}{e}
\]

Let us define $\trExp{\_}{\_}$ case by case. In the case of literals, we get:

\[
\trExp{\_}{c} \equiv (\addE~\istau{\taulit{\hat{c}}};\addE~\tauvalue{\taulit{\hat{c}}} = \hat{c}, \taulit{\hat{c}})
\]

where $\tau$ is the type of the literal, which can be determined at compile time
since it is a literal, and $\hat{c}$ is the constant in the underlying logic represented 
by that literal. For example, the Elixir term $\mathbf{2}$ corresponds to the 
actual number $2 \in \mathbb{Z}$, so $\hat{\mathbf{2}} = 2$.

In the case of variables, we get:

\[
\trExp{\_}{x} \equiv (\skipE, \hat{x})
\]

It returns the logic variable $\hat{x}$ corresponding to the L1 variable $x$. 
No L0 expression is generated.

The L0 expressions generated by a tuple correspond to the ones generated by each 
component, the projection function for each one and its tuple size function. Its 
translated term is a specific tuple constructor for its size $n$ applied to its 
translated term components:

\[
\begin{array}{l}
\trExp{\gamma}{\{e_1, \ldots, e_n\}} \equiv (\epsilon_1; \dots; \epsilon_n;\epsilon;\epsilon'_1; \dots; \epsilon'_n, t)\\
\qquad \textbf{where } \forall i \in \{1..n\}. (\epsilon_i, t_i) = \trExp{\gamma}{e_i}\\
\qquad \phantom{\textbf{where }} t = \mathit{n\mbox{-}tuple}(t_1, \ldots, t_n)\\
\qquad \phantom{\textbf{where }} \epsilon = \addE~\istuple{t};\addE~\mathit{tuple\mbox{-}size}(t) = n\\
\qquad \phantom{\textbf{where }} \forall i \in \{1..n\}. \epsilon'_i = \addE~\mathit{elem}(t, i) = t_i\\
\end{array}
\]

The translation for lists is defined recursively, with the empty list 
as the base case. The generated L0 expressions set the corresponding heads and tails 
for the generated list terms, and it does not require the second
argument for the list constructor to be a list:

\[
\begin{array}{l}
\trExp{\_}{[]} \equiv (
  \skipE, \mathit{nil}
)\\
\trExp{\gamma}{[e_1~|~e_2]} \equiv (\epsilon_1;\epsilon_2;\epsilon, t)\\
  \qquad \textbf{where } (\epsilon_1, t_1) = \trExp{\gamma}{e_1}\\ 
  \qquad \phantom{\textbf{where }} (\epsilon_2, t_2) = \trExp{\gamma}{e_2}\\ 
  \qquad \phantom{\textbf{where }} t = \mathit{cons}(t_1, t_2)\\ 
  \qquad \phantom{\textbf{where }} \epsilon = \left[
    \begin{array}{l}
      \addE~\isnelist{t};\\
      \addE~\mathit{hd}(t) = t_1;\\
      \addE~\mathit{tl}(t) = t_2\\
    \end{array}
  \right]
\end{array}
\]

The most complex case is that of function application:

\[
\begin{array}{l}
\trExp{\gamma}{f(e_1, \ldots, e_n)} \equiv (\epsilon_1; \dots; \epsilon_n; \epsilon; \overline{\epsilon_\sigma}^{\sigma \in \Spec{f}}, \widehat{f}(t_1, \ldots, t_n))\\
\qquad \textbf{where } \forall i \in \{1..n\}. (\epsilon_i, t_i) = \trExp{\gamma}{e_i}\\
\qquad \phantom{\textbf{where }} \epsilon = \left[ 
  \begin{array}{l}
  \whenUnsatW~\gamma;\addE~\neg \bigvee_{\sigma \in \Spec{f}}\Pre{\sigma}(t_1\ldots,t_n)\\
  \quad\doW~\skipE\\
  \quad\elseW~\failE\\
  \end{array}
  \right]\\
\qquad \phantom{\textbf{where }} \forall \sigma \in \Spec{f} \textrm{ such that } \sigma = (\varphi_\sigma(x_1\ldots,x_n), \psi_\sigma(x_1, \ldots, x_n)).\\
\qquad\qquad\qquad\qquad \epsilon_\sigma = \left[ 
\begin{array}{l}
\whenUnsatW~\gamma;\addE~\neg \varphi_\sigma(t_1\ldots,t_n)~\doW\\
\quad\addE~\varphi_\sigma(t_1\ldots,t_n);\\
\quad\addE~\psi_\sigma(t_1,\ldots,t_n)\\
\elseW~\skipE\\
\end{array}
\right] \\
\end{array}
\]

Firstly, we generate the L0 expression $\epsilon_i$ corresponding to each 
argument $e_i$, and its corresponding uninterpreted term $t_i$. Then, for 
each pre/post-condition pair of the specification of the function being
applied, we generate code that checks whether the precondition holds and, 
in case it does, we assert both the precondition and postcondition. Finally,
we also check and fail if none of the preconditions hold.

We distinguish the cases of logical connectives from function application
because of their specific short-circuit semantics in Elixir:

\[
\begin{array}{l}
\trExp{\gamma}{\andE{e_1}{e_2}} \equiv (\epsilon, t)\\
\qquad \textbf{where }(\epsilon_1, t_1) = \trExp{\gamma}{e_1}\\
\qquad \phantom{\textbf{where }}(\epsilon_2, t_2) = \trExp{\gamma'}{e_2}\\
\qquad\qquad\qquad\qquad \gamma' = \gamma;\addE~\booleanvalue{t_1}\\
\qquad\qquad\qquad\qquad t = \widehat{\mathbf{and}}(t_1, t_2)\\
\qquad\qquad\qquad\qquad \epsilon = \left[ 
\begin{array}{l}
\epsilon_1;\\
\whenUnsatW~\gamma;\addE~\neg\isboolean{t_1}~\doW\\
\quad\whenUnsatW~\gamma;\addE~\booleanvalue{t_1}~\doW\\
\quad\quad\addE~t = \false\\
\quad\elseW\\
\quad\quad\epsilon_2;\\
\quad\quad\whenUnsatW~\gamma;\addE~\neg\booleanvalue{t_1}~\doW\\
\quad\quad\quad\addE~t = t_2\\
\quad\quad\elseW~\whenUnsatW~\gamma';\addE~\neg\isboolean{t_2}~\doW\\
\quad\quad\quad\addE~\isboolean{t};\\
\quad\quad\quad\addE~\booleanvalue{t} = \booleanvalue{t_1} \\
\quad\phantom{\quad\quad\addE~\booleanvalue{t}} \land \booleanvalue{t_2}\\
\quad\quad\elseW~\failE\\
\elseW~\failE
\end{array}
\right] \\
\end{array}
\]

In the translation for an $\mathbf{and}$ expression, we firstly check if 
the term to the left is a boolean. Then, on the one hand, if it is known to be always 
$\false$, the resulting term is $\false$. On the other hand, if it 
is known to be always $\true$, the resulting term is the right one 
regardless of its type. Note that this right term has been translated with the 
knowledge that the left one is $\true$. If the value of the 
left term is not exactly known at this point, we check if the right term 
is a boolean, again with the knowledge that the left one is $\true$,
and translate the whole expression into the underlying logical conjunction.

The translation corresponding to $\mathbf{or}$ is analogous:

\[
\begin{array}{l}
\trExp{\gamma}{\orE{e_1}{e_2}} \equiv (\epsilon, t)\\
\qquad \textbf{where }(\epsilon_1, t_1) = \trExp{\gamma}{e_1}\\
\qquad \phantom{\textbf{where }}(\epsilon_2, t_2) = \trExp{\gamma'}{e_2}\\
\qquad\qquad\qquad\qquad \gamma' = \gamma;\addE~\neg\booleanvalue{t_1}\\
\qquad\qquad\qquad\qquad t = \widehat{\mathbf{or}}(t_1, t_2)\\
\qquad\qquad\qquad\qquad \epsilon = \left[ 
\begin{array}{l}
\epsilon_1;\\
\whenUnsatW~\gamma;\addE~\neg\isboolean{t_1}~\doW\\
\quad\whenUnsatW~\gamma;\addE~\neg\booleanvalue{t_1}~\doW\\
\quad\quad\addE~t = \true\\
\quad\elseW\\
\quad\quad\epsilon_2;\\
\quad\quad\whenUnsatW~\gamma;\addE~\booleanvalue{t_1}~\doW\\
\quad\quad\quad\addE~t = t_2\\
\quad\quad\elseW~\whenUnsatW~\gamma';\addE~\neg\isboolean{t_2}~\doW\\
\quad\quad\quad\addE~\isboolean{t};\\
\quad\quad\quad\addE~\booleanvalue{t} = \booleanvalue{t_1} \\
\quad\phantom{\quad\quad\addE~\booleanvalue{t}} \lor \booleanvalue{t_2}\\
\quad\quad\elseW~\failE\\
\elseW~\failE
\end{array}
\right] \\
\end{array}
\]

Now we move on to L1 statements. The following ones are translated in a 
quite straightforward way:

\[
\begin{array}{l}
\trStm{\skipE} \equiv \skipE\\[1em]
\trStm{\blockE~S} \equiv \localE~\trStm{S}\\[1em]
\trStm{\havocE~x} \equiv \declareConstE{\widehat{x}}\\[1em]
\trStm{S_1;S_2} \equiv \trStm{S_1};\trStm{S_2}\\
\qquad \textbf{where }(\epsilon_1, t_1) = \trExps{e_1}\\
\qquad \phantom{\textbf{where }}(\epsilon_2, t_2) = \trExps{e_2}
\end{array}
\]

In the case of $\assumeE$, we generate the expression $\epsilon$ that 
corresponds to the expression being assumed and its uninterpreted term $t$. 
We ensure that the term $t$ actually denotes a boolean value and, in this 
case, we assert that this boolean value is $\true$:

\[
\trStm{\assumeE~e} \equiv 
\left[\begin{array}{l}
\epsilon;\\
\whenUnsatW~\addE~\neg\isboolean{t}\\
\quad\doW~\addE~\booleanvalue{t}\\
\quad\elseW~\failE\\
\end{array}\right]
\qquad \textbf{where}~(\epsilon, t) = \trExps{e}
\]

In the case of $\assertE$, we also generate the expression $\epsilon$ that 
corresponds to the expression being assumed and its uninterpreted term $t$. 
We ensure that the term $t$ actually denotes a boolean value and also that 
its boolean value is $\true$:
\[
\trStm{\assertE~e} \equiv
\left[\begin{array}{l}
\epsilon;\\
\whenUnsatW~\addE~\neg\isboolean{t}\\
\quad\doW~\skipE\\
\quad\elseW~\failE;\\
\whenUnsatW~\addE~\neg\booleanvalue{t}\\
\quad\doW~\addE~\booleanvalue{t}\\
\quad\elseW~\failE\\
\end{array}\right]
\qquad \textbf{where}~(\epsilon, t) = \trExps{e}
\]

Finally, the $\unfoldE$ statement relies on having an user-provided definition
for the involved function body as a parameterized L1 expression, and also 
user-provided pre/post-condition pair specifications in terms of L1 expressions:

\[
\begin{array}{l}
\trStm{\unfoldE~f(e_1, \dots, e_n)} \equiv \epsilon;\overline{\epsilon_\sigma}^{\sigma \in \UserSpec{f}}\\
\qquad \textbf{where } \epsilon = \trStm{\assumeE~f(e_1, \dots, e_n)~\texttt{===}~\Body{f}(e_1, \dots, e_n)}\\
\qquad \phantom{\textbf{where }} \forall \sigma \in \UserSpec{f} \textrm{ such that } \sigma = (p_\sigma(e_1\ldots,e_n), q_\sigma(e_1, \ldots, e_n)).\\
\qquad\qquad\qquad\qquad (\epsilon_p, t_p) = \trExps{p_\sigma(e_1\ldots,e_n)}\\
\qquad\qquad\qquad\qquad \epsilon_\sigma = \left[ 
\begin{array}{l}
\epsilon_p;\\
\whenUnsatW~\addE~\neg\isboolean{t_p}~\doW\\
\quad\whenUnsatW~\addE~\neg\booleanvalue{t_p}~\doW\\
\quad\quad\trStm{\assumeE~p_\sigma(e_1\ldots,e_n)};\\
\quad\quad\trStm{\assumeE~q_\sigma(e_1\ldots,e_n)}\\
\quad\elseW~\skipE\\
\elseW~\skipE
\end{array}
\right] \\
\end{array}
\]

\subsection{Term size modelling}

\textcolor{red}{TODO}

Intended to be used for reasoning about termination.

We also provide the following axioms in order to reason about
term sizes:

\[
\begin{array}{l}
\termSize{\mathit{nil}} = 1\\
\forall x. \isinteger{x} \Rightarrow \termSize{x} = 1\\
\forall x. \isboolean{x} \Rightarrow \termSize{x} = 1\\
\forall x. \isnelist{x} \Rightarrow \termSize{x} = 1 + \termSize{\mathit{hd}(x)} + \termSize{\mathit{tl}(x)}\\
\forall x. \istuple{x} \Rightarrow \forall i. i >= 0 \wedge i < \mathit{tuple\mbox{-}size}(x) \Rightarrow \termSize{\mathit{elem}(x, i)} < \termSize{x}\\
\end{array}
\]

They are based on the types and will be useful for reasoning about termination.

\section{Implementation}

\textcolor{red}{TODO}

Show examples in a separate section before this one if possible.

Mention relevant facts of the development process and implementation details.
