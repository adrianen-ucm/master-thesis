\chapter{The L1 Intermediate Representation}
\label{cap:intermediateRepresentation}

\chapterquote{This process of using tools you built yesterday to help build bigger tools today is called abstraction, and it is the most powerful force I know of in the universe}{Sandy Maguire}

In this chapter, we develop an \gls{ir} for verification that is intended to be 
particularly suitable for languages like Elixir. We start by providing its
formal syntax, its translation into L0 expressions and the built-in prelude that
we have defined for modelling the Elixir semantics.  Then, we show some examples
with our current implementation and give an overview of its details.

\section{Syntax}

We denote by $\Sigma^{1} = \{ \texttt{===}, \texttt{<=}, \texttt{>=}, 
\texttt{+}, \texttt{-}, \dots \}$ the set of operators and functions allowed in
L1. We assume that for every element $f \in \Sigma^{1}$ there is an 
uninterpreted function symbol in $\Sigma^{0}$, which will be denoted by 
$\widehat{f}$. If $f$ has arity $n$, its corresponding function symbol 
$\widehat{f}$ will have sort $\Term \times \stackrel{n}{\dots} \times \Term
\rightarrow \Term$.

Let us define the syntax of L1 expressions and statements:

\[
\begin{array}{rcll}
\Exp{1} \ni e & ::= & c & \{ \textrm{literal} \}\\
& | & x & \{ \textrm{variable} \}\\
& | & \andE{e_1}{e_2} & \{ \textrm{conjunction} \}\\
& | & \orE{e_1}{e_2} & \{ \textrm{disjunction} \}\\
& | & [] & \{ \textrm{empty list} \}\\
& | & [e_1~|~e_2] & \{ \textrm{list cons cell} \}\\
& | & \{e_1, \ldots, e_n\} & \{ \textrm{tuple} \}\\
& | & f(e_1, \ldots, e_n) & \{ \textrm{function or operator application} \}\\

\\
\Stm \ni S & ::= & \skipE & \{ \textrm{do nothing} \} \\
 & | & \blockE~S & \{ \textrm{local scoped evaluation} \}\\
 & | & \havocE~x & \{ \textrm{variable declaration} \}\\
 & | & S_1;S_2 & \{ \textrm{sequential evaluation} \}\\
 & | & \assumeE~e & \{ \textrm{assume a formula} \}\\
 & | & \assertE~e & \{ \textrm{assert a formula} \}\\
\end{array}
\]

We also assume that for every function symbol $f \in \Sigma^{1}$ of arity $n$
there is an overloaded specification expressed in terms of L0 formulae. Here the
word \emph{overloaded} means that there could be many pre/post-condition pairs
for each function. For example, equality can be specified as follows:

\[
\begin{array}{l}
\{ \isinteger{x} \wedge \isinteger{y} \}\\
x~\texttt{===}~y\\
\{ \booleanvalue{\widehat{===}(x, y)} \Leftrightarrow \integervalue{x} = \integervalue{y} \}\\
\\
\{ \isboolean{x} \wedge \isboolean{y} \}\\
x~\texttt{===}~y\\
\{ \booleanvalue{\widehat{===}(x, y)} \Leftrightarrow \booleanvalue{x} = \booleanvalue{y} \}\\
\\
\vdots
\\
\\
\{\true\}\\
x~\texttt{===}~y\\
\{ \isboolean{\widehat{===}(x, y)} \wedge \booleanvalue{\widehat{===}(x, y)} \Leftrightarrow (x = y) \}\\
\\
\end{array}
\]

Here $\widehat{===}$ is the uninterpreted symbol in $\Sigma^{0}$ corresponding 
to Elixir's strict equality operator $\texttt{===} \in \Sigma^{1}$. We write the
former in prefix form in order to highlight the fact that it is an uninterpreted
function symbol in the logic. On the contrary, the $=$, $\Leftrightarrow$,
$\wedge$ in the specification above are actual connectives and operators of the
underlying logic.

We denote by $\sigma_1, \ldots, \sigma_m$ the specifications of a function $f
\in \Sigma^{1}$. Each one is a pair $(\varphi(x_1, \ldots, x_n), \psi(x_1,
\ldots, x_n))$, where the $x_i$ variables denote the parameters of the function.
We also denote by $\Spec{f}$ the set of specifications of $f$. They will be
built-in into the system in order to model the Elixir semantics.

\section{Semantics}

In this section, we show the translation process from L1 statements and
expressions into L0 expressions. Also, we show how the most relevant function
specifications to model the Elixir semantics are built-in into the system, but
we allow the user to introduce their own function definitions and specifications
in terms of L1 expressions.

\subsection{Built-in declarations}

During the translation of L1 statements and expressions into L0 expressions, we 
require some defined sorts, constants and functions in SMT-LIB. Every representation
of an L1 expression has sort $\Term$ in the underlying logic:

\begin{verbatim}
; The sort Term
(declare-sort Term 0)
\end{verbatim}

Terms can be of a given type:

\begin{verbatim}
; The sort Type
(declare-sort Type 0)

; Modelled types
(declare-const int Type)
(declare-const bool Type)
(declare-const tuple Type)
(declare-const nonempty_list Type)

; All of them are different
(assert (distinct int bool))
(assert (distinct int tuple))
(assert (distinct int nonempty_list))
(assert (distinct bool tuple))
(assert (distinct bool nonempty_list))
(assert (distinct tuple nonempty_list))

; Type membership checking predicates
(declare-fun type (Term) Type)
(define-fun is_integer ((x Term)) Bool (= (type x) int))
(define-fun is_boolean ((x Term)) Bool (= (type x) bool))
(define-fun is_tuple ((x Term)) Bool (= (type x) tuple))

(define-fun is_nonempty_list ((x Term)) Bool (and
  (distinct x nil) 
  (= (type x) nonempty_list)
))

(define-fun is_list ((x Term)) Bool (or 
  (= x nil) 
  (= (type x) nonempty_list)
))
\end{verbatim}

We also need a way to introduce and eliminate $\Term$s:

\begin{verbatim}
; Introduce literal values as corresponding Term
(declare-fun integer_lit (Int) Term)
(declare-fun boolean_lit (Bool) Term)

; In the case of boolean connectives, their value constructors
(declare-fun term_and (Term Term) Term)
(declare-fun term_or (Term Term) Term)

; In the case of lists, their value constructors
(declare-fun nil () Term)
(declare-fun cons (Term Term) Term)

; In the case of tuples, they are declared dynamically as
; (declare-fun tuple_n (Term ... Term) Term)

; The value of a Term of a given type in the underlying logic
(declare-fun integer_val (Term) Int)
(declare-fun boolean_val (Term) Bool)

; In the case of lists and tuples, its decomposition
(declare-fun hd (Term) Term)
(declare-fun tl (Term) Term)
(declare-fun tuple_size (Term) Int)
(declare-fun elem (Term Int) Term)
\end{verbatim}

These SMT-LIB commands will be executed at the beginning of our verification
process to initialize the solver.

\subsection{Translation into L0}
\label{sec:translationL0}

When it comes to assign a meaning to L1 statements and expressions, as this is
an \gls{ir}, we translate them into L0 expressions to verify them. For this, we
shall define two functions:

\[
\begin{array}{ll}
\trExp{\_}{\_} : & \Exp{0} \times \Exp{1} \rightarrow \Exp{0} \times \mathbb{T}\\
\trStm{\_} : & \Stm \rightarrow \Exp{0}
\end{array}
\]

Given an L1 expression $e$, the application $\trExp{\gamma}{e}$ returns a tuple 
$(\epsilon, t)$, in which $\epsilon$ is an L0 expression that models the
semantics of $e$, and $t$ is the term in the underlying logic that will be used
to refer to the result of $e$. The L0 expression $\gamma$ models those facts
that are known by the time $e$ is evaluated and is needed to handle the short
circuit-based semantics of $\mathbf{and}$ and $\mathbf{or}$. We are going to
omit this $\gamma$ parameter when there are no such facts to be assumed:

\[
\trExps{e} \equiv \trExp{\skipE}{e}
\]

Let us define $\trExp{\_}{\_}$ case by case. In the case of literals, we get:

\[
\trExp{\gamma}{c} \equiv (\addE~\istau{\taulit{\hat{c}}};\addE~\tauvalue{\taulit{\hat{c}}} = \hat{c}, \taulit{\hat{c}})
\]

where $\tau$ is the type of the literal, which can be determined at compile time
since it is a literal, and $\hat{c}$ is the constant in the underlying logic
represented by that literal. For example, the Elixir term $\mathbf{2}$
corresponds to the actual number $2 \in \mathbb{Z}$, so $\hat{\mathbf{2}} = 2$.

In the case of variables, we get:

\[
\trExp{\gamma}{x} \equiv (\skipE, \hat{x})
\]

where $\hat{x}$ is the logic variable corresponding to the L1 variable $x$. No
L0 expression is generated.

The L0 expressions generated by a tuple correspond to the ones generated by each
component, a series of facts that define the projection function for each one,
and assertions on the type of the expression and the size of the tuple. Its
translated term is a specific tuple constructor for its size $n$ applied to its
translated term components:

\[
\begin{array}{l}
\trExp{\gamma}{\{e_1, \ldots, e_n\}} \equiv (\epsilon_1; \dots; \epsilon_n;\epsilon;\epsilon'_1; \dots; \epsilon'_n, t)\\
\qquad \textbf{where } \forall i \in \{1..n\}. (\epsilon_i, t_i) = \trExp{\gamma}{e_i}\\
\qquad \phantom{\textbf{where }} t = \mathit{n\mbox{-}tuple}(t_1, \ldots, t_n)\\
\qquad \phantom{\textbf{where }} \epsilon = \addE~\istuple{t};\addE~\mathit{tuple\mbox{-}size}(t) = n\\
\qquad \phantom{\textbf{where }} \forall i \in \{1..n\}. \epsilon'_i = \addE~\mathit{elem}(t, i) = t_i\\
\end{array}
\]

The translation for lists is defined recursively, with the empty list as the
base case. The generated L0 expressions define the corresponding heads and tails
for the generated list terms, and it does not require the second argument for
the list constructor to be a list:

\[
\begin{array}{l}
\trExp{\_}{[]} \equiv (
  \skipE, \mathit{nil}
)\\
\trExp{\gamma}{[e_1~|~e_2]} \equiv (\epsilon_1;\epsilon_2;\epsilon, t)\\
  \qquad \textbf{where } (\epsilon_1, t_1) = \trExp{\gamma}{e_1}\\ 
  \qquad \phantom{\textbf{where }} (\epsilon_2, t_2) = \trExp{\gamma}{e_2}\\ 
  \qquad \phantom{\textbf{where }} t = \mathit{cons}(t_1, t_2)\\ 
  \qquad \phantom{\textbf{where }} \epsilon = \left[
    \begin{array}{l}
      \addE~\isnelist{t};\\
      \addE~\mathit{hd}(t) = t_1;\\
      \addE~\mathit{tl}(t) = t_2\\
    \end{array}
  \right]
\end{array}
\]

A more complex case is that of function application:

\[
\begin{array}{l}
\trExp{\gamma}{f(e_1, \ldots, e_n)} \equiv (\epsilon_1; \dots; \epsilon_n; \epsilon; \overline{\epsilon_\sigma}^{\sigma \in \Spec{f}}, \widehat{f}(t_1, \ldots, t_n))\\
\qquad \textbf{where } \forall i \in \{1..n\}. (\epsilon_i, t_i) = \trExp{\gamma}{e_i}\\
\qquad \phantom{\textbf{where }} \epsilon = \left[ 
  \begin{array}{l}
  \whenUnsatW~\gamma;\addE~\neg \bigvee_{\sigma \in \Spec{f}}\Pre{\sigma}(t_1\ldots,t_n)\\
  \quad\doW~\skipE\\
  \quad\elseW~\failE\\
  \end{array}
  \right]\\
\qquad \phantom{\textbf{where }} \forall \sigma \in \Spec{f} \textrm{ such that } \sigma = (\varphi_\sigma(x_1\ldots,x_n), \psi_\sigma(x_1, \ldots, x_n)).\\
\qquad\qquad\qquad\qquad \epsilon_\sigma = \left[ 
\begin{array}{l}
\whenUnsatW~\gamma;\addE~\neg \varphi_\sigma(t_1\ldots,t_n)~\doW\\
\quad\addE~\varphi_\sigma(t_1\ldots,t_n);\\
\quad\addE~\psi_\sigma(t_1,\ldots,t_n)\\
\elseW~\skipE\\
\end{array}
\right] \\
\end{array}
\]

Firstly, we generate the L0 expression $\epsilon_i$ corresponding to each 
argument $e_i$, and its corresponding uninterpreted term $t_i$. Then, we check 
that at least one preconditions holds and, finally, for each pre/post-condition
pair of the specification of the function being applied, we generate code that
checks whether the precondition holds and, in case it does, we assert both the
precondition and postcondition.

We distinguish the cases of logical connectives from function application
because of their specific short-circuit semantics in Elixir:

\[
\begin{array}{l}
\trExp{\gamma}{\andE{e_1}{e_2}} \equiv (\epsilon, t)\\
\qquad \textbf{where }(\epsilon_1, t_1) = \trExp{\gamma}{e_1}\\
\qquad \phantom{\textbf{where }}(\epsilon_2, t_2) = \trExp{\gamma'}{e_2}\\
\qquad\qquad\qquad\qquad \gamma' = \gamma;\addE~\booleanvalue{t_1}\\
\qquad\qquad\qquad\qquad t = \widehat{\mathbf{and}}(t_1, t_2)\\
\qquad\qquad\qquad\qquad \epsilon = \left[ 
\begin{array}{l}
\epsilon_1;\\
\whenUnsatW~\gamma;\addE~\neg\isboolean{t_1}~\doW\\
\quad\whenUnsatW~\gamma;\addE~\booleanvalue{t_1}~\doW\\
\quad\quad\addE~\isboolean{t};\\
\quad\quad\addE~\neg\booleanvalue{t};\\
\quad\quad\addE~\neg\booleanvalue{t_1}\\
\quad\elseW\\
\quad\quad\epsilon_2;\\
\quad\quad\whenUnsatW~\gamma;\addE~\neg\booleanvalue{t_1}~\doW\\
\quad\quad\quad\addE~\booleanvalue{t_1}\\
\quad\quad\quad\addE~t = t_2\\
\quad\quad\elseW~\whenUnsatW~\gamma';\addE~\neg\isboolean{t_2}~\doW\\
\quad\quad\quad\addE~\isboolean{t};\\
\quad\quad\quad\addE~\booleanvalue{t} = \\
\quad\quad\quad\quad(\booleanvalue{t_1} \land \booleanvalue{t_2})\\
\quad\quad\elseW~\failE\\
\elseW~\failE
\end{array}
\right] \\
\end{array}
\]

In the translation for an $\mathbf{and}$ expression, we firstly check if $t_1$,
the term in the underlying logic that results of translating $e_1$, is boolean.
Then, on the one hand, if it is known to be always $\false$, the resulting term
is $\false$.  On the other hand, if it is known to be always $\true$, the
resulting term is $t_2$ regardless of its type. Note that $e_2$ has been
translated with the knowledge that $t_1$ is $\true$. If the value of $t_1$ is
not exactly known at this point, we check if $t_2$ is a boolean, again with the
knowledge that $t_1$ is $\true$, and translate the whole expression into the
underlying logical conjunction.

The translation corresponding to $\mathbf{or}$ is analogous:

\[
\begin{array}{l}
\trExp{\gamma}{\orE{e_1}{e_2}} \equiv (\epsilon, t)\\
\qquad \textbf{where }(\epsilon_1, t_1) = \trExp{\gamma}{e_1}\\
\qquad \phantom{\textbf{where }}(\epsilon_2, t_2) = \trExp{\gamma'}{e_2}\\
\qquad\qquad\qquad\qquad \gamma' = \gamma;\addE~\neg\booleanvalue{t_1}\\
\qquad\qquad\qquad\qquad t = \widehat{\mathbf{or}}(t_1, t_2)\\
\qquad\qquad\qquad\qquad \epsilon = \left[ 
\begin{array}{l}
\epsilon_1;\\
\whenUnsatW~\gamma;\addE~\neg\isboolean{t_1}~\doW\\
\quad\whenUnsatW~\gamma;\addE~\neg\booleanvalue{t_1}~\doW\\
\quad\quad\addE~\isboolean{t};\\
\quad\quad\addE~\booleanvalue{t};\\
\quad\quad\addE~\booleanvalue{t_1}\\
\quad\elseW\\
\quad\quad\epsilon_2;\\
\quad\quad\whenUnsatW~\gamma;\addE~\booleanvalue{t_1}~\doW\\
\quad\quad\quad\addE~\neg\booleanvalue{t_1}\\
\quad\quad\quad\addE~t = t_2\\
\quad\quad\elseW~\whenUnsatW~\gamma';\addE~\neg\isboolean{t_2}~\doW\\
\quad\quad\quad\addE~\isboolean{t};\\
\quad\quad\quad\addE~\booleanvalue{t} = \\
\quad\quad\quad\quad(\booleanvalue{t_1} \lor \booleanvalue{t_2})\\
\quad\quad\elseW~\failE\\
\elseW~\failE
\end{array}
\right] \\
\end{array}
\]

Now we move on to L1 statements. The following ones are translated in a 
quite straightforward way:

\[
\begin{array}{l}
\trStm{\skipE} \equiv \skipE\\[1em]
\trStm{\blockE~S} \equiv \localE~\trStm{S}\\[1em]
\trStm{\havocE~x} \equiv \declareE{\widehat{x}}\\[1em]
\trStm{S_1;S_2} \equiv \trStm{S_1};\trStm{S_2}\\
\qquad \textbf{where }(\epsilon_1, t_1) = \trExps{e_1}\\
\qquad \phantom{\textbf{where }}(\epsilon_2, t_2) = \trExps{e_2}
\end{array}
\]

In the case of $\assumeE$, we generate the expression $\epsilon$ that 
corresponds to the expression being assumed and its uninterpreted term $t$. 
We ensure that the term $t$ actually denotes a boolean value and, in this 
case, we assert that this boolean value is $\true$:

\[
\trStm{\assumeE~e} \equiv 
\left[\begin{array}{l}
\epsilon;\\
\whenUnsatW~\addE~\neg\isboolean{t}\\
\quad\doW~\addE~\booleanvalue{t}\\
\quad\elseW~\failE\\
\end{array}\right]
\qquad \textbf{where}~(\epsilon, t) = \trExps{e}
\]

In the case of $\assertE$, we also generate the expression $\epsilon$ that 
corresponds to the expression being assumed and its uninterpreted term $t$, but
now we check whether the term $t$ actually denotes a boolean value and also that
its boolean value is $\true$:

\[
\trStm{\assertE~e} \equiv
\left[\begin{array}{l}
\epsilon;\\
\whenUnsatW~\addE~\neg\isboolean{t}\\
\quad\doW~\skipE\\
\quad\elseW~\failE;\\
\whenUnsatW~\addE~\neg\booleanvalue{t}\\
\quad\doW~\addE~\booleanvalue{t}\\
\quad\elseW~\failE\\
\end{array}\right]
\qquad \textbf{where}~(\epsilon, t) = \trExps{e}
\]

\subsection{Built-in specifications}

In order to allow L1 expressions to model the semantics of built-in Elixir
functions and operators, the corresponding uninterpreted functions must be
declared in SMT-LIB with sort $\Term \times \stackrel{n}{\dots} \times \Term
\rightarrow \Term$, and our system must provide their corresponding built-in
specifications. We have explored some of them which are explained in this
section.

For integer arithmetic, the specification of $\texttt{+}$ can be defined as

\[
\begin{array}{l}
\{ \isinteger{x} \wedge \isinteger{y} \}\\
x~\texttt{+}~y\\
\{ \isinteger{\widehat{+}(x, y)} \wedge 
  \integervalue{\widehat{+}(x, y)} = \integervalue{x} + \integervalue{y} \}
\end{array}
\]

and it could be extended if we modelled other numeric types such as
\verb|float|. The binary operators $\texttt{*}$ and $\texttt{-}$ and are
specified similarly, and the unary version of $\texttt{-}$ can be specified as
follows:

\[
\begin{array}{l}
\{ \isinteger{x} \}\\
\texttt{-}~x\\
\{ \isinteger{\widehat{-}(x)} \wedge 
  \integervalue{\widehat{-}(x)} = -\integervalue{x} \}
\end{array}
\]

Similarly, the Elixir boolean negation can be specified as:

\[
\begin{array}{l}
\{ \isboolean{x} \}\\
\mathit{not}(x)\\
\{ \isboolean{\widehat{\mathit{not}}(x)} \wedge 
  \booleanvalue{\widehat{\mathit{not}}(x)} \Leftrightarrow \neg\booleanvalue{x} \}
\end{array}
\]

We have only provided the comparison for integer terms as

\[
\begin{array}{l}
\{ \isinteger{x} \wedge \isinteger{y} \}\\
x~\texttt{<}~y\\
\{ \isboolean{\widehat{<}(x, y)} \wedge 
  \booleanvalue{\widehat{<}(x, y)} \Leftrightarrow \integervalue{x} < \integervalue{y} \}
\end{array}
\]

and it is in the same way for $\texttt{>}$, $\texttt{<=}$ and $\texttt{>=}$. An
improvement would be to extend this for any term, including lists and tuples.

Term equality can be specified as follows:

\[
\begin{array}{l}
\{ \isinteger{x} \wedge \isinteger{y} \}\\
x~\texttt{===}~y\\
\{ \booleanvalue{\widehat{===}(x, y)} \Leftrightarrow \integervalue{x} = \integervalue{y} \}\\
\\
\{ \isboolean{x} \wedge \isboolean{y} \}\\
x~\texttt{===}~y\\
\{ \booleanvalue{\widehat{===}(x, y)} \Leftrightarrow \booleanvalue{x} = \booleanvalue{y} \}\\
\\
\{ \islist{x} \wedge \islist{y} \}\\
x~\texttt{===}~y\\
\{ \booleanvalue{\widehat{===}(x, y)} \Leftrightarrow 
  (x = \mathit{nil} \wedge y = \mathit{nil})
  \vee (x \neq \mathit{nil} \wedge y \neq \mathit{nil} \wedge\\
  \quad\quad\mathit{hd}(x) = \mathit{hd}(y) \land \mathit{tl}(x) = \mathit{tl}(y)) \}\\
\\
\{ \istuple{x} \wedge \istuple{y} \wedge \mathit{tuple\mbox{-}size}(x) = \mathit{tuple\mbox{-}size}(y) \}\\
x~\texttt{===}~y\\
\{ \booleanvalue{\widehat{===}(x, y)} \Leftrightarrow 
  (\forall i. i >= 0 \wedge i < \mathit{tuple\mbox{-}size}(x) \Rightarrow \mathit{elem}(x, i) = \mathit{elem}(y, i)) \}\\
\\
\{ \istuple{x} \wedge \istuple{y} \wedge \mathit{tuple\mbox{-}size}(x) \neq \mathit{tuple\mbox{-}size}(y) \}\\
x~\texttt{===}~y\\
\{ \neg\booleanvalue{\widehat{===}(x, y)} \}\\
\\
\{\true\}\\
x~\texttt{===}~y\\
\{ \isboolean{\widehat{===}(x, y)} \wedge \booleanvalue{\widehat{===}(x, y)} \Leftrightarrow (x = y) \}\\
\\
\end{array}
\]

and it is also similar for $\texttt{!==}$.

The $\mathit{tuple\mbox{-}size}$ and $\mathit{elem}$ functions can be specified directly 
in terms of the built-in declarations used during the translation:

\[
\begin{array}{l}
\{ \istuple{x} \}\\
\mathit{tuple\mbox{-}size}(x)\\
\{ \isinteger{\widehat{\mathit{tuple\mbox{-}size}}(x)} \wedge 
  \integervalue{\widehat{\mathit{tuple\mbox{-}size}}(x)} = \mathit{tuple\mbox{-}size}(x) \}
\end{array}
\]

\[
\begin{array}{l}
\{ \istuple{x} \wedge \isinteger{i} \wedge \integervalue{i} >= 0 \wedge \integervalue{i} < \mathit{tuple\mbox{-}size}(x) \}\\
\mathit{elem}(x, i)\\
\{ \widehat{\mathit{elem}}(x, i) = \mathit{elem}(x, \integervalue{i}) \}
\end{array}
\]

The same can be applied to the $\mathit{hd}$ function

\[
\begin{array}{l}
\{ \isnelist{x} \}\\
\mathit{hd}(x)\\
\{ \widehat{\mathit{hd}}(x) = \mathit{hd}(x) \}
\end{array}
\]

and it is similar for $\mathit{tl}$. Note that, in these last examples, the L1
function that is being specified is not the same as the one mentioned in the
postcondition, which is a built-in L0 function, although we have used the same
name.

The type-checking functions can also be specified directly with the built-in
declared L0 functions:

\[
\begin{array}{l}
\{ \true \}\\
\isinteger{x}\\
\{ \isboolean{\widehat{\isinteger{x}}} \wedge 
  \booleanvalue{\widehat{\isinteger{x}}} \Leftrightarrow \isinteger{x} \}
\end{array}
\]

and it is similar for the remaining types.

If some Elixir operator cannot be modelled with function specifications like the
ones shown in this section, they can be defined as L1 expressions with its own 
translation into L0, as we did to model the short-circuit semantics of the 
\verb|and| and \verb|or| operators.

\subsection{Term size modelling}
\label{sec:termsize}

Although total correctness is beyond the scope of this work, checking function
termination is subject of future work, mainly when verifying recursive Elixir
function definitions. With that purpose, we have considered an uninterpreted
function to assign an integer value to every $\Term$

\begin{verbatim}
(declare-fun term_size (Term) Int)
\end{verbatim}

and a set of axioms based on their types

\[
\begin{array}{l}
\termSize{\mathit{nil}} = 1\\
\forall x. \isinteger{x} \Rightarrow \termSize{x} = 1\\
\forall x. \isboolean{x} \Rightarrow \termSize{x} = 1\\
\forall x. \isnelist{x} \Rightarrow \termSize{x} = 1 + \termSize{\mathit{hd}(x)} + \termSize{\mathit{tl}(x)}\\
\forall x. \istuple{x} \Rightarrow \forall i. i >= 0 \wedge i < \mathit{tuple\mbox{-}size}(x) \Rightarrow \termSize{\mathit{elem}(x, i)} < \termSize{x}\\
\end{array}
\]

Note that this is a proposal. We have not worked on this for the moment beyond
some small experiments within our implementation.

\section{Implementation}
\label{ir:l1implementation}

In this section, we present our implementation for the L1 verification \gls{ir}
in Elixir.

\subsection{Overview}
\label{ir:dslexample}

We have implemented a \gls{dsl} in Elixir to write and verify L1 programs. The
macro \verb|with_local_env/1| provides a default environment with a fresh Z3
connection that is injected as the first argument of its inner \gls{dsl} macros
(i.e. \verb|assert/2|, \verb|havoc/2|, etc.) and is closed once all of them have
been evaluated. As in our \acrshort{smt} binding, \verb|with_local_env/1| is a
convenient wrapper around \verb|with_env/2|, which allows one to provide a
custom or reused connection and does not close it automatically.

Here are some examples that succeed to verify according to the Elixir behavior 
explained in Section \ref{prelim:elixir}:

\begin{lstlisting}[language=elixir,numbers=none,frame=none]
import Boogiex

with_local_env do
  assert 4 - 2 === 6 - 4
  assert (false or 2) === 2
  assert 3 > 2 and 1 <= 1
  assert elem({1, 2, 3}, 0) === 1
  assert [1 | [2 | [3 | []]]] === [1, 2, 3]
  assert true or true + true

  havoc x
  assert x === x
  assert not (x !== x)

  block do
    assume x === 2
    assert is_integer(x), "This should not fail"
  end

  assert is_integer(x), "This should fail"

  havoc a
  havoc b
  havoc c

  assume is_integer(a) and is_integer(b)
  assume is_integer(a) and is_integer(b) and is_integer(c)

  assume a === b
  assume b === c
  assert a === c

  assert false, "This should fail"
end
\end{lstlisting}

The package also exposes the corresponding translation functions between
\gls{dsl}s for implementing other tools on top of them.

\subsection{Details}
\label{ir:details}

We are going to explain our implementation in a schematic way in order to give
its main idea. 

First, we implement a function to translate Elixir \gls{ast} corresponding to L1
expressions, together with an assumption given as an L0 expression, into the L0
term that it represents and another L0 expression that models its semantics:

\begin{lstlisting}[language=elixir,numbers=none,frame=none]
@spec translate_l1_exp(L0Exp.ast, L1Exp.ast) 
  :: {L0Exp.ast, L0Exp.ast}
\end{lstlisting}

Its definition syntax matches pretty closely the formal version, as in this case
for nonempty lists:

\begin{lstlisting}[language=elixir,numbers=none,frame=none]
def translate_l1_exp(assumption, [{:|, _, [h, t]}]) do
  {head, head_sem} = translate_l1_exp(assumption, h)
  {tail, tail_sem} = translate_l1_exp(assumption, t)
  term = quote(do: :cons.(unquote(head), unquote(tail)))

  {
    term,
    quote do
      unquote(head_sem)
      unquote(tail_sem)
      add :is_nonempty_list.(unquote(term))
      add :hd.(unquote(term)) == unquote(head)
      add :tl.(unquote(term)) == unquote(tail)
    end
  }
end
\end{lstlisting}

This translation relies on a state of tuple constructors that are declared on 
demand. We also provide a mechanism to indicate the context of the program in
order to report helpful error messages, but we have omitted such details for the
sake of brevity. 

Then, we also implement a function to translate Elixir \gls{ast} corresponding
to L1 statements into L0 code:

\begin{lstlisting}[language=elixir,numbers=none,frame=none]
@spec translate_l1_stm(L1Stm.ast) :: L0Exp.ast
\end{lstlisting}

It also matches closely the formal version, as in this case for the 
\verb|assert| statement:

\begin{lstlisting}[language=elixir,numbers=none,frame=none]
def translate_l1_stm([{:assert, _, [f]}]) do
  {term, term_sem} = translate_l1_exp(nil, f)

  quote do
    when_unsat add !:is_boolean.(unquote(term)) do
    else
      fail
    end

    when_unsat add !:boolean_val.(unquote(term)) do
      add :boolean_val.(unquote(term))
    else
      fail
    end
  end
end
\end{lstlisting}

Finally, this allows us to implement a public \gls{api} for the package which
defines the corresponding macros from the example at Section
\ref{ir:dslexample}. These macros translate the L1 \gls{dsl} into L0 and
evaluate the resulting L0 expression as in Section \ref{integ:l0implementation}.
A verification function for L1 statements can be implemented in terms of it as
follows but, in contrast to the one presented in Section
\ref{integ:l0implementation}, our current implementation returns verification
error reports instead of raising an exception and stopping the whole process:

\begin{lstlisting}[language=elixir,numbers=none,frame=none]
@spec verify_l1(Env.t(), L1Stm.ast()) :: [term()]
def verify_l1(env, s) do
  L0Exp.eval(
    env,
    translate_l1_stm(s)
  )
end
\end{lstlisting}
