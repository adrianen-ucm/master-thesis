\chapter{State of the Art}
\label{cap:stateOfTheArt}

Our main goal is to provide a static verification mechanism to allow Elixir
programmers to write formal specifications and prove the conformance of their 
code in regard to these specifications. 

In this section, we will discuss the current approaches and tools for that purpose
in general, and then which are the current approaches to prove or disprove correctness
in Elixir.

\section{Program verification}

A usual approach for program verification is to transform code and specifications into
an intermediate representation, such as Boogie, and then a theorem prover
tries to prove its verification conditions, where their validity implies the correctness
under consideration \citep{Boogie2}.

In general, a theorem prover may not be able to reach a required proof, although it may exist,
and human intervention can be necessary in the form of interacting with an interface, transforming
the code to be verified or adding information to help the prover [reference].

\subsection{Dafny}

Dafny is a programming language that provides features for program verification and 
covers several programming paradigms, such as imperative, functional and \gls{oop} \citep{DafnyManual}. 
It was created at Microsoft Research and is currently being developed with the support of Amazon.

The following is an example to specify and verify the implementation of a function to return 
the maximum of three integer numbers:

\begin{verbatim}
method max3(x: int, y: int, z: int) returns (m: int)
  ensures m == x || m == y || m == z
  ensures m >= x && m >= y && m >= y
{
  if x > y && x > z { return x; }
  if y > z { return y; }
  return z;
}
\end{verbatim}

Although the above example may seem simple, Dafny can also handle more advanced topics such as recursion 
and loops by means of induction and loop invariants respectively. Its verified code can be translated into 
other programming languages, mainly C\#, to be executed. 

Our project is greatly inspired by this system, although our aim is to embed program verification features 
into an existing programming language instead of translate verified code into an executable version.

\subsection{Intermediate representations for verification}

\gls{ir}s tend to arise in compiler technologies, with goals like perform analyses, optimizations or 
portability \citep{FormalizingLLVMIR}. A richer language can be translated into a simpler or more focused
one, the \gls{ir}, which can also be the translation target for other languages.

A known technology which provides a platform-independent \gls{ir} intended to be executed in 
different platforms, and a toolchain to work with it, is LLVM \citep{LLVM}.

Programming languages such as Java and Erlang also have as a compilation target a bytecode \gls{ir} 
corresponding to their virtual machines, JVM and BEAM respectively. These are also compilation targets 
for other programming languages such as Kotlin and Scala for the JVM and Elixir for the BEAM virtual
machine. Also, WebAssembly is an \gls{ir} intended be executable at native speed in web browsers [reference].

For building verification tools, we are interested on \gls{ir}s that are focused in capturing the 
intended verification notions and are suitable to be transformed into input for a theorem prover.
This last one will try to provide a proof for the verification.

\textcolor{red}{TODO:} discutir algunas

\begin{itemize}
  \item Boogie
  \item Why3 / WhyML
  \item Viper
  \item CAVI-ART
  \item CHC (Constrained Horn Clauses)
  \item Rule-based representation (COSTA group)
\end{itemize}

\subsection{SMT solvers}

Z3, CVC4, MATHSAT, Yices... SMT-LIB

\section{Correctness in Elixir}

\subsection{Propery-based testing}

Propery-based testing: Proper, QuickCheck (excheck)

\subsection{Erlang Verification Tool}

Erlang Verification Tool (Lars-Åke Freudlund)
