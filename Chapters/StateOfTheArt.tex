\chapter{State of the Art}
\label{cap:stateOfTheArt}

\chapterquote{Don't you want to be a person who inspires others the way you were inspired by something?}{Fredrik Thordendal}

Our main goal is to provide a static verification mechanism to allow Elixir
programmers to write formal specifications and prove the conformance of their 
code in regard to these specifications. In this section we will discuss the
current approaches and tools for that purpose in general, and then which are the
current approaches to prove or disprove correctness properties in Elixir.

\section{Program verification}

A usual approach for program verification is based on transforming code and
specifications into an \gls{ir}, such as Boogie, and then a theorem prover tries
to prove the verification conditions that are extracted from the \gls{ir}. The
validity of these conditions implies the correctness of the program under
consideration \citep{Boogie2}.

In general, a theorem prover may not be able to reach a required proof, although
such a proof may exist, and human intervention can be necessary in the form of
interacting with an interface, transforming the code to be verified or adding
information to help the prover.

\subsection{Dafny}

Dafny is a programming language that provides features for program verification
and covers several programming paradigms, such as imperative, functional and
\gls{oop} \citep{DafnyManual}.  It was created at Microsoft Research and is
currently being developed with the support of Amazon.

The following is an example to specify and verify the implementation of a
function to return the maximum of three integer numbers:

\begin{lstlisting}[language=dafny,numbers=none,frame=none]
method max3(x: int, y: int, z: int) returns (m: int)
  ensures m == x || m == y || m == z
  ensures m >= x && m >= y && m >= z
{
  if x > y && x > z { return x; }
  if y > z { return y; }
  return z;
}
\end{lstlisting}

Although the above example may seem simple, Dafny can also handle more advanced
topics such as recursion and loops by means of induction and loop invariants
respectively. Once verified, the code can be translated, erasing everything that
is related only to its verification, into other programming languages such as
C\# so that it is compiled and executed. 

Our project is greatly inspired by this system, although our aim is to embed
program verification features into an existing programming language, namely
Elixir, instead. Dafny code is also translated into a verification \gls{ir}, 
named Boogie, for which verification conditions are intended to be discharged by
the Z3 theorem prover. These different levels of abstraction correspond to
separate tools, each one with its own parser, whereas in our project we pretend
to embed all of them in Elixir.

\subsection{Intermediate representations}

\gls{ir}s tend to arise in compiler technologies, with goals like performing
analyses, optimizations or portability \citep{FormalizingLLVMIR}. A richer
language can be translated into a simpler or more focused one, the \gls{ir},
which can also be the translation target for other languages.

A known technology which provides a platform-independent \gls{ir} intended to be
executed in different platforms, and a toolchain to work with it, is \gls{llvm}
\citep{LLVM}. This toolchain is more focused on optimization and static
analysis, but there are also symbolic execution tools for \gls{llvm} programs,
like in \cite{cadar2008klee}, which has been used in program verification for
the Rust programming language in \citep{lindner2018no}.

Programming languages such as Java and Erlang also have as a compilation target
a bytecode \gls{ir} corresponding to their virtual machines, \gls{jvm} 
\citep{lindholm2013java} and BEAM \citep{stenman2015erlang} respectively. These
are also compilation targets for other programming languages such as Kotlin and
Scala for the \gls{jvm} and Elixir for the BEAM virtual machine.  Also,
WebAssembly is an \gls{ir} intended to be executable at native speed in web
browsers \citep{wasm}.

\subsection{Intermediate representations for verification}

For building verification tools, we are interested in \gls{ir}s that are focused
on capturing the intended verification notions and are suitable to be
transformed into verification conditions that can be sent to a theorem prover.
The latter will try to provide a proof for the correctness of the program under
verification.

Apart from Boogie, which is the verification \gls{ir} used in Dafny and offers 
features to model a wide range of programming paradigms, there are other ones as
\gls{viper}, a suite of tools which provides an intermediate verification
language with the same name and allows reasoning about the program state using
\textit{permissions} or \textit{ownership}. It allows implementing verification
techniques for sequential and concurrent programs with mutable state
\citep{viper}. Viper has also been used for program verification in Rust 
\citep{astrauskas2022prusti}.

Also, Why3 \citep{Why3} is a platform for deductive program verification that
provides a language for specification and programming, WhyML, that can be used
as an \gls{ir} for the verification of C, Java, Ada, or to obtain automated
correct-by-construction OCaml programs \citep{Why3Manual}.

Other approaches use logic programs as an \gls{ir} throughout its regular
compilation process, as in \cite{horn}, and other ones put an effort into being
a suitable target for modelling faithfully the semantics of different
programming languages, as in \cite{caviart}. The latter has been used in a
project that is related with the one presented in this document, because they
use S-expressions as a textual representation of their \gls{ir} and process them
with Common Lisp, also using Lisp macros, in order to provide different
semantics, one of them for generating verification conditions \citep{cavilisp}.

\subsection{SMT solvers}

At the lowest level of the verification process, a theorem prover will try to
obtain a proof for some verification conditions. \acrshort{smt} solvers in
particular are gaining popularity for this task, and some current options are Z3
from Microsoft \citep{moura2008z3}, CVC4 \citep{barrett2011cvc4}, MathSAT
\citep{bozzano2005mathsat} and Yices \citep{dutertre2006yices}. 

There is also an international initiative, called SMT-LIB, aimed at facilitating
research and development in \gls{smt} \citep{smtLibStandard}. We will discuss 
\gls{smt}, SMT-LIB and Z3 in the Chapter \ref{cap:preliminaries}, as these are
tools that are going to be used in our project.

\section{Program verification in Elixir}

The current approaches for proving or disproving the correctness of Elixir 
programs are in general inherited from Erlang, and they succeed more in 
detecting faults rather than in proving correctness.

\subsection{Dynamic approaches}

Erlang provides several libraries for property-based testing such as
\gls{proper} \citep{proper} and Erlang QuickCheck \citep{quickcheck}. Many of
them are offered for Elixir through wrapper packages as \verb|PropCheck| for
\gls{proper}.

There are also packages implemented completely in Elixir, such as
\verb|StreamData| and \verb|TypeCheck|. This last one tries to take profit of
Elixir type specifications in order to automatically provide data generators.

In this case, their goal is to disprove the correctness of the program by
finding counterexamples for specified properties. Tools of this kind show the
presence of errors rather than the absence of them.

\subsection{Static approaches}
\label{sec:static}

Elixir allows annotating the intended types for function parameters and return
values, so that a tool named \gls{dialyzer} \citep{Dialyzer} can perform static
analysis on them. A type discrepancy detected by the tool implies that the
program is bound to fail when executed, but the absence of such discrepancies
does not guarantee type safety, and it does neither prove that the program is
correct, which is our aim in this project.

There have been other attempts in providing tools that offer correctness proofs
for Erlang programs, as in \cite{erlangveri}, where they allow the programmer to
specify correctness requirements with inductive and compositional reasoning, and
also with reasoning about side-effect-free code. They also offer a graphical
tool to interact with the theorem prover, but conclude that more effort is
needed in order to reduce the required human intervention and to make its
verification method practically useful.
