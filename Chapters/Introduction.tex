\chapter{Introduction}
\label{cap:introduction}

\textcolor{red}{TODO}: introduction. At some point explain the general idea, 
including a diagram, and include a guide referencing the chapters.

\section{Motivation}

\textcolor{red}{TODO}

\section{Goals}

The main goal of this project is to use the Elixir metaprogramming capabilities
through macros to implement a code verification system for the Elixir
programming language itself, without requiring us to modify its compiler or to
implement a parser.

Our system will rely on a verification \gls{ir} and the use of \acrshort{smt}
solvers for its verification.

\subsection{Sub-goals}

In order to achieve the main goal, we have proposed several possible sub-goals.
One of them is to integrate \acrshort{smt} solvers in Elixir with a \gls{dsl} of
macros like in the following draft:

Another one is to develop a verification \gls{ir} to express Elixir terms and
its dynamically typed nature:

Then, we must define a translation from this \gls{ir} into the \acrshort{smt}
solver \gls{dsl} for its verification and, finally, we must also provide a
mechanism to translate a subset of the Elixir programming language into the
verification \gls{ir}.

\subsection{Non-goals}

We have left some points as possible future work. The first one is that we are
going to deal only with sequential Elixir programs and not with concurrent ones.
Even for the sequential Elixir part we are going to support only a small subset 
to start with.

Also, we are going to deal only with partial verification for the moment and not
to verify or reason about termination. Nevertheless, we will discuss some ideas 
regarding it when introducing user-defined verification functions and how the 
system allows to unfold their invocations.  

\section{Work plan}

As a work plan, we will follow the sub-goals in the same order in which we have
mentioned them, preceded by a training period for acquiring the required
knowledge and practice with the Elixir programming language.
