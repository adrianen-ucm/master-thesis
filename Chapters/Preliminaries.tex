\chapter{Preliminaries}
\label{cap:preliminaries}

This chapter introduces some required topics and tools that are relevant to
our project. 

On the one hand, Elixir is the programming language that is the verification 
subject of this document and, at the same time, the one in which our 
implementation has been coded. 

On the other hand, our verification system relies on the \gls{smt} problem and
its encodings in SMT-LIB, a standard language and interface to interact with 
theorem provers such as Z3.

\section{Elixir}

Elixir is a general-purpose programming language that runs on the Erlang
Virtual Machine, where also the Erlang programming language runs. Both
languages share some features, like their actor-based concurrency model, and
have a native capability to interoperate between them. Although Elixir is
younger than Erlang, this has allowed the former to be part of an ecosystem
which has been developed across more than three decades.

We have chosen such a programming language for this research because,
first, it is a modern programming language ready to be used in the
industry [referenciar]. Second, it has the unusual property in formal 
verification to be dynamically typed, but its functional programming principles
will make its reasoning easier. Finally, its metaprogramming capabilities will
allow us to extend it according to our needs without requiring to modify its 
compiler.

\subsection{General description}

In this section, we introduce the basic concepts and constructs of sequential
Elixir programming. [El objetivo de esto es, a parte de presentar el lenguaje,
el de exponer cómo se comporta el subconjunto del lenguaje que hemos estudiado
por el momento para su verificación].

[mencionar iex para los snippets]

\subsubsection{Value types}

As usual, one of its built-in value types is the \verb|integer|, for which
arithmetic operators behave as expected:

\begin{lstlisting}[language=elixir,numbers=none,frame=none]
  iex> (2 + 2) * 5
  20
  iex> -1
  -1
  iex> 1 / 0
  ** (ArithmeticError)
\end{lstlisting}

The \verb|boolean| type is also present, but its operators have some
worth to mention semantics when involving non-\verb|boolean| types:

\begin{lstlisting}[language=elixir,numbers=none,frame=none]
  iex> true and 2
  2
  iex> 2 and true
  ** (BadBooleanError)
  iex> false or 2
  2
  iex> 2 or false
  ** (BadBooleanError)
  true
\end{lstlisting}

And also with respect to short-circuit evaluation:

\begin{lstlisting}[language=elixir,numbers=none,frame=none]
  iex> false and 1 / 0
  false
  iex> true or 1 / 0
  true
\end{lstlisting}

Some built-in Elixir functions allow to check if a term is of a given type in 
terms of a \verb|boolean| result:

\begin{lstlisting}[language=elixir,numbers=none,frame=none]
  iex> is_boolean(true)
  true
  iex> is_boolean(2)
  false
  iex> is_integer(2)
  true
\end{lstlisting}

Equality and comparison operators also evaluate to \verb|boolean| values and
allow to mix types:

\begin{lstlisting}[language=elixir,numbers=none,frame=none]
  iex> 2 === true
  false
  iex> 2 === 2
  true
  iex> 2 > 1
  true
\end{lstlisting}

\subsubsection{Collection types and pattern matching}

One of the simplest built-in collection types in Elixir is the inductive 
\verb|list|, which can be .

\begin{lstlisting}[language=elixir,numbers=none,frame=none]
  iex> [1, 2, false]
  [1, 2, false]
  iex> [1, 2 | [false]]
  [1, 2, false]
  iex> [1 | [2 | [false | []]]]
  [1, 2, false]
  iex> [1 | [2 | [3 | 4]]]
  [1, 2, 3 | 4] # An improper list
\end{lstlisting}

\begin{lstlisting}[language=elixir,numbers=none,frame=none]
  iex> hd([1, 2, false])
  1
  iex> tl([1, 2, false])
  [2, false]
  iex> hd([])
  ** (ArgumentError)
  iex> tl([])
  ** (ArgumentError)
  iex> Enum.all?(                 # All
    [[], [1, 2], [1, 2 | false]], # of these
    &is_list/1                    # are lists
  )
  true
\end{lstlisting}

\subsubsection{Function definitions}

Comment what has to do with pattern matching

Multiple bodies, name+arity, and recursion

\subsubsection{Blocks, assignments and control flow}

Comment what has to do with pattern matching

\subsection{Macros}

Describe, this will be our way to extend Elixir for code verification.

\subsection{Interoperability}

Different ways (ports, \gls{nif}s), for the \acrshort{smt} solver integration.

\section{Satisfiability Modulo Theories}

The \gls{smt} problem consists of checking whether
a given logical formula is satisfiable within a specific theory
\citep{smtLibStandard}. This allows to define theories in which the \gls{smt} problem
is decidable and, moreover, to design efficient algorithms specialized in 
solving this problem for a theory.

Poner un ejemplo referenciado de una teoría con \gls{smt} decidible y algún 
algoritmo eficiente.

\subsection{SMT-LIB}

SMT-LIB is an initiative which tries to provide a common interface to interact
with \acrshort{smt} solvers. It defines an agnostic standard language with a Lisp-like
syntax to both configure a solver, manage it, encode an \gls{smt} problem 
instance and query for solutions.

General description (many sorted, citado) and example, show the subset of commands that we are going to use

\[
\begin{array}{rcll}
\langle\textit{ command }\rangle & ::= & \texttt{( assert } \langle\textit{ term }\rangle\texttt{ ) }\\
& | & \texttt{( check-sat )}\\
& | & \texttt{( pop } \langle\textit{ numeral }\rangle \texttt{ ) }\\
& | & \texttt{( push } \langle\textit{ numeral }\rangle \texttt{ ) }\\
& | & \texttt{( declare-sort } \langle\textit{ symbol }\rangle~\langle\textit{ numeral }\rangle\texttt{ ) }\\
& | & \texttt{( declare-const } \langle\textit{ symbol }\rangle~\langle\textit{ sort }\rangle\texttt{ ) }\\
& | & \texttt{( declare-fun } \langle\textit{ symbol }\rangle \texttt{ ( }\langle\textit{ symbol }\rangle^* \texttt{ ) }\langle\textit{ sort }\rangle\texttt{ ) }\\
& | & \texttt{( define-fun } \langle\textit{ function\_def }\rangle\texttt{ ) }\\
\end{array}
\]

Poner un problema sencillo (el mismo de la sección anterior)

\subsection{Z3}

One of the \gls{smt} solvers that implements the SMT-LIB standard is the Z3 theorem
prover from Microsoft Research.

Decir porque lo hemos elegido y que aun así hemos intentado utilizarlo como 
interprete SMT-LIB para no depender completamente de el.

Note that there may exist subtle discrepancies when implementing the SMT-LIB 
standard. For example, while trying to use the \verb|echo| command as a delimiter 
for the solver responses as suggested in the standard document, we have found that Z3 
does not include the surrounding double-quotes when it prints back the provided 
string literal, which is specified in the standard. This behavior may add confusion 
because the \verb|echo| command is the only one that responds with a string and, in 
Z3, the \texttt{( echo "sat") } response is exactly the same as the 
\texttt{( check-sat )} one when it returns \texttt{sat}, so a command response can be 
confused with the string used to delimit command responses.
