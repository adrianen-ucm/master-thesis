\chapter{Preliminaries}
\label{cap:preliminaries}

This chapter introduces some required topics and tools that are a basis to our
project.

On the one hand, Elixir is the programming language that is the verification
subject of this document and, at the same time, the one in which our
implementation has been coded.

On the other hand, our verification system relies on the \gls{smt} problem and
its encoding in SMT-LIB, a standard language and interface to interact with
theorem provers such as Z3.

\section{Elixir}

Elixir is a general-purpose programming language that runs on the Erlang Virtual
Machine, also called BEAM, where also programs written in the Erlang language
run. Both of them share some features, like their actor-based concurrency model,
and have a native capability to interoperate between them. Although Elixir is
younger than Erlang, this has allowed the former to be part of an ecosystem
which has been developed across more than three decades.

We have chosen such a programming language for this research because, first, it
is a modern programming language ready to be used in the industry.  Second, it
has the unusual property in formal verification to be dynamically typed, but its
functional programming principles will make it easier to reason about. Finally,
its metaprogramming capabilities will allow us to extend it according to our
needs without requiring us to modify its compiler.

\subsection{General description}

In this section, we introduce the basic concepts and constructs of sequential
programming in Elixir. Our aim is to show only the behavior of the language
subset that is studied later in this document for its verification, and also its
metaprogramming mechanism based on macros, on top of which our proposed
verification system has been implemented.

The following examples will be shown in the Elixir \gls{repl}, called
\verb|iex|, where \verb|iex>| represents its default prompt and an introduced
expression is followed by the result such that it evaluates:

\begin{lstlisting}[language=elixir,numbers=none,frame=none]
iex> "Hello world"
"Hello world"
\end{lstlisting}

\subsubsection{Value types}

As usual, one of the core value types in Elixir is \verb|integer|, for which
arithmetic operators behave as expected:

\begin{lstlisting}[language=elixir,numbers=none,frame=none]
iex> (2 + 2) * 5
20
iex> -1
-1
iex> 1 / 0
** (ArithmeticError)
\end{lstlisting}

The \verb|boolean| value type is also at its core, but it is worth mentioning
the semantics of its operators when involving non-\verb|boolean| types, and also
with respect to short-circuit evaluation:

\begin{lstlisting}[language=elixir,numbers=none,frame=none]
iex> true and 2 # Evaluates to the second argument
2
iex> 2 and true # Requires the first one to be a boolean
** (BadBooleanError)
iex> false and 1 / 0 # Does not evaluate the second argument
false
\end{lstlisting}

Some built-in Elixir functions allow checking if a given value is of a given
type by returning a \verb|boolean| result:

\begin{lstlisting}[language=elixir,numbers=none,frame=none]
iex> is_boolean(true)
true
iex> is_boolean(2)
false
iex> is_integer(2)
true
\end{lstlisting}

Equality and comparison operators also evaluate to \verb|boolean| values and
allow mixing types:

\begin{lstlisting}[language=elixir,numbers=none,frame=none]
iex> 2 === 2
true
iex> 2 === true
false
iex> 2 !== true
true
iex> 2 > 1
true
iex> 2 < true
true
\end{lstlisting}

The \verb|===|, \verb|!==|, \verb|and| and \verb|or| operators are the so-called
\textit{strict} version of their respective counterparts \verb|==|, \verb|!=|,
\verb|&&| and \verb+||+, but we are not going to deal with them for the moment.

Also, \verb|boolean| values are in fact a special case for \verb|atom| values,
but we are not going to deal with that value type for the moment in this
project.

\subsubsection{Collection types}

One of the simplest built-in collection types in Elixir is the inductive 
\verb|list|, which consists of nested cons cells (i.e. pairs) and can be written
in different ways:

\begin{lstlisting}[language=elixir,numbers=none,frame=none]
iex> [] # The empty list 
[] 
iex> [3 | []] # A cons cell
[3]
iex> [1 | [2 | [3 | []]]] # Nested cons cells
[1, 2, 3]
iex> [1, 2, 3] # Syntactic sugar
[1, 2, 3]
iex> [1, 2 | [3]] # Mixing sugared and desugared syntax
[1, 2, 3]
\end{lstlisting}

It is not required for the \verb|list| elements to be of the same type (i.e.
heterogeneous lists are allowed), and improper lists (i.e. those that do not
have an empty list as the second element in the deepest cons cell) are also
allowed \citep{ElixirDocs}:

\begin{lstlisting}[language=elixir,numbers=none,frame=none]
iex> [1, 2, false] # An heterogeneous list
[1, 2, false]
iex> [1 | [2 | 3]] # An improper list
[1, 2 | 3]
\end{lstlisting}

Functions in Elixir are commonly referred by its name and arity. The \verb|hd/1|
and \verb|tl/1| built-in functions for lists allow us to respectively obtain the
first and second components of a cons cell:

\begin{lstlisting}[language=elixir,numbers=none,frame=none]
iex> hd([1, 2, false])
1
iex> tl([1, 2, false])
[2, false]
iex> hd([])
** (ArgumentError)
iex> tl([])
** (ArgumentError)
\end{lstlisting}

There is also a function for checking the \verb|list| type membership. Consider
the following code to apply the \verb|is_list/1| function to several provided 
lists and return the conjunction of its results:

\begin{lstlisting}[language=elixir,numbers=none,frame=none]
iex> Enum.all?(                 # Are all
  [[], [1, 2], [1, 2 | false]], # of these
  &is_list/1                    # lists?
)
true                            # Yes
\end{lstlisting}

Another core collection type in Elixir is the \verb|tuple|, which also does not
restrict its elements to be of the same type:

\begin{lstlisting}[language=elixir,numbers=none,frame=none]
iex> {} # The empty tuple
{}
iex> {1, false, {3, 4}, []}
{1, false, {3, 4}, []}
\end{lstlisting}

Tuples have a size, which can be retrieved with the \verb|tuple_size/1|
function, and each \verb|tuple| component can also be retrieved with the
\verb|elem/2| function by specifying its position with a zero-based index:

\begin{lstlisting}[language=elixir,numbers=none,frame=none]
iex> tuple_size({1, 2, 3})
3
iex> elem({1, 2, 3}, 0)
1
iex> elem({1, 2, 3}, 2)
3
iex> elem({1, 2, 3}, 3)
** (ArgumentError)
\end{lstlisting}

In this case, the \verb|tuple| type membership checking function is
\verb|is_tuple/1|. Usually, the components of a collection such as a \verb|list|
or a \verb|tuple| are obtained by means of pattern matching, that is explained
in the following section.

\subsubsection{Blocks, pattern matching and control flow}

Elixir expressions can be evaluated sequentially by gathering them inside a
\verb|block|, delimited by a semicolon or a line break:

\begin{lstlisting}[language=elixir,numbers=none,frame=none]
iex> 2 + 1;5 === 5;false
false # Evaluates to the result of its last expressions
iex> (
        2 + 1
        5 === 5
        false
     )
false
\end{lstlisting}

The expressions inside a \verb|block| that are not the last one tend to perform
side effects, such as binding values to variable names with the \verb|match|
operator \verb|=|. Note that these bindings are not locally scoped inside
\verb|block|s and, in contrast to Erlang, variable bindings can be overridden:

\begin{lstlisting}[language=elixir,numbers=none,frame=none]
iex> (z = 2, 4)
4
iex> z
2
iex> z = 3
3
\end{lstlisting}

This operator also allows performing pattern matching, which destructures 
expressions according to patterns in order to check for a given shape and 
bind subexpressions to variable names. They are particularly useful for 
dealing with collection value types:

\begin{lstlisting}[language=elixir,numbers=none,frame=none]
iex> {x, 3} = {2, 3}
{2, 3}
iex> x
2
iex> {x, 3} = {2, 4}
** (MatchError)
iex> [h | t = [_, 3]] = [1, 2, 3] # A nested match
[1, 2, 3]
iex> h
1
iex> t
[2, 3]
\end{lstlisting}

Regarding control flow, although Elixir provides usual constructs such as 
\verb|if|, one of the most general ones is \verb|case|. It is evaluated to the
first branch that matches the pattern and is compliant with a guard expression
if specified, and this is the only branch in the \verb|case| that is evaluated:

\begin{lstlisting}[language=elixir,numbers=none,frame=none]
iex> case {1, 2, 3} do 
        {}                           -> 1 / 0
        {1, x, 3} when is_integer(x) -> x + 1
        {1, 2, 3}                    -> false
     end
3
iex> x
** (CompileError) # The case bindings are local
iex> case 2 do 
        false -> 3 
     end 
** (CaseClauseError) # No pattern matches the expression
\end{lstlisting}

Guards have a restricted syntax, allowing for example comparison, boolean 
negation, conjunction and disjunction, and type checking for values.

\subsubsection{Function definitions}

A named function, identified by its name and arity, can be defined inside a
\verb|module| with different body definitions and different patterns and guards
for its arguments:

\begin{lstlisting}[language=elixir,numbers=none,frame=none]
defmodule Example do 
  def fact(0) do 
    1
  end

  def fact(n) when is_integer(n) and n > 0 do 
    n * fact(n - 1) # Recursion is allowed
  end
end
\end{lstlisting}

The rules that determine which clause is applied is are the same as in
\verb|case| expressions, so function definitions can also express control flow
\citep{programmingElixir}.

\subsubsection{Type specifications}
\label{prelim:typespecs}

Although Elixir is dynamically typed, it has a system to annotate the intended
types for functions and a tool to perform a static analysis on them, which is 
called dialyzer \citep{Dialyzer}. We will use these specifications together with
function identifiers to outline the ideas behind our implementations along this
document.

A function type specification can be defined as follows: 

\begin{lstlisting}[language=elixir,numbers=none,frame=none]
@spec function_name(type_1, type_2, ... type_n) :: return_type
\end{lstlisting}

Types can be defined by means of composing other types with constructs such as
the \verb+|+ operator, which denotes the union of types:

\begin{lstlisting}[language=elixir,numbers=none,frame=none]
@type tuple_or_nat :: tuple | non_neg_integer
\end{lstlisting}

\subsection{Macros}

Because of its metaprogramming capabilities based on macros, Elixir is a
suitable language for implementing \gls{dsl}s \citep{metaprogrammingElixir}.
This will allow us to extend it without requiring us to modify the Elixir
compiler or implement a parser.

The main construct for this purpose is \verb|defmacro| which, as a curiosity, is
declared in the \verb|Kernel| module of Elixir in terms of itself due to a
bootstrapping process:

\begin{lstlisting}[language=elixir,numbers=none,frame=none]
  defmacro defmacro(call, expr) do
    ...
\end{lstlisting}

The argument values for a macro are Elixir \gls{ast}s, and its return value must
also be a valid Elixir \gls{ast} that will replace the macro invocation at
compile-time. The resulting code may also contain other macro calls that will be
expanded recursively.

By using type specifications, the \gls{ast} type for Elixir expressions is
defined in the \verb|Macro| module as

\begin{lstlisting}[language=elixir,numbers=none,frame=none]
@type ast ::
  atom
  | number
  | [ast]
  | {ast, ast}
  | ast_expr

@type ast_expr :: {ast_expr | atom, metadata, atom | [ast]}
\end{lstlisting}

where \verb|ast_expr| represents a function invocation when the first component
is the function name, and the third one its arguments. We can obtain the
\gls{ast} corresponding to an Elixir expression with the \verb|quote/1| macro:

\begin{lstlisting}[language=elixir,numbers=none,frame=none]
iex> quote do 
        1 + 2
     end
{:+, [context: Elixir, import: Kernel], [1, 2]}
\end{lstlisting}

\verb|quote/1| is the main construct to transform the input \gls{ast} into new
\gls{ast} when defining a macro, together with \verb|unquote/1| to interpolate
expressions inside a quoted one:

\begin{lstlisting}[language=elixir,numbers=none,frame=none]
defmacro sum_into_product({:+, _, [x, y]}) do
  quote do
    unquote(x) * unquote(y)
  end
end
\end{lstlisting}

Elixir also offers several advanced constructs to deal with macros, such as
\verb|unquote_splicing/1| to interpolate an \gls{ast} list as the arguments of a
function invocation, and \verb|escape/1| to escape \gls{ast} data (e.g. it
allows to introduce some given \gls{ast} as data instead of as code in a macro
definition):

\begin{lstlisting}[language=elixir,numbers=none,frame=none]
iex> quote do
        hello(unquote_splicing([2, 3]))
     end
{:hello, [], [2, 3]}
iex> ast = quote(do: 2 + 2)
iex> quote do
        unquote(ast)
     end
{:+, [context: Elixir, import: Kernel], [2, 2]}
iex> quote do 
        unquote(Macro.escape(ast))
     end
{:{}, [], [:+, [context: Elixir, import: Kernel], [2, 2]]}
\end{lstlisting}

\subsection{Interoperability}

Elixir offers several ways to interoperate with processes or libraries that are
external to the Erlang Virtual Machine, apart from conventional \gls{io} based
mechanisms. We are interested in these features due to the integration of an
\acrshort{smt} solver in Elixir, which will surely be an external process.

One of these ways is \gls{nif}s, which allows loading and calling libraries
implemented in other programming languages such as C. When using this system, it
is important to know that a crash in a \gls{nif} brings the Erlang Virtual 
Machine down too \citep{ErlangDocs}.

A safer approach is to launch an external process managed by the Erlang Virtual
Machine and communicate with it by means of message passing, which in Elixir is
provided by a mechanism called \textit{ports}:

\begin{lstlisting}[language=elixir,numbers=none,frame=none]
port = Port.open({:spawn, "cat"}, [:binary])
iex> send(port, {self(), {:command, "hello"}})
iex> flush()
# Received from the process
{#Port<0.1444>, {:data, "hello"}} 
send(port, {self(), :close})
\end{lstlisting}

The underlying implementation makes the communication through \verb|stdin| and 
\verb|stdout|, but this is abstracted under the message passing \gls{api}.

A known drawback of this mechanism is that, if the Erlang Virtual Machine
crashes after having launched a long-running process, then its \verb|stdin| and
\verb|stdout| channels will be closed, but it won't be automatically terminated.
This depends on how the specific process behaves when its communication channels
are closed \citep{ElixirDocs}.

\section{Satisfiability Modulo Theories}

The \gls{smt} problem consists in checking whether a given logical formula is
satisfiable within a specific theory \citep{smtLibStandard}. This allows a 
theorem prover to define theories in which the \gls{smt} problem is decidable
and, moreover, to design efficient algorithms specialized in solving this
problem for a theory.

\textcolor{red}{TODO}: reference and comment a theory with smt decidable and some 
efficient algorithm.

\subsection{SMT-LIB}
\label{prelim:smtlib}

SMT-LIB is an initiative which tries to provide a common interface to interact
with \acrshort{smt} solvers. It defines a solver-agnostic standard language with
a Lisp-like syntax to configure a solver, manage it, encode an \gls{smt} problem
instance and query for solutions.

\textcolor{red}{TODO}: general description, many sorted and references. 
Show the subset of commands and responses that we are going to use.

\[
\begin{array}{rcll}
\langle\textit{ command }\rangle & ::= & \texttt{( assert } \langle\textit{ term }\rangle\texttt{ ) }\\
& | & \texttt{( check-sat )}\\
& | & \texttt{( pop } \langle\textit{ numeral }\rangle \texttt{ ) }\\
& | & \texttt{( push } \langle\textit{ numeral }\rangle \texttt{ ) }\\
& | & \texttt{( declare-sort } \langle\textit{ symbol }\rangle~\langle\textit{ numeral }\rangle\texttt{ ) }\\
& | & \texttt{( declare-const } \langle\textit{ symbol }\rangle~\langle\textit{ sort }\rangle\texttt{ ) }\\
& | & \texttt{( declare-fun } \langle\textit{ symbol }\rangle \texttt{ ( }\langle\textit{ symbol }\rangle^* \texttt{ ) }\langle\textit{ sort }\rangle\texttt{ ) }\\
\end{array}
\]

\[
\begin{array}{rcll}
\langle\textit{ general\_response }\rangle & ::= & \texttt{success}\\
& | & \texttt{unsupported}\\
& | & \texttt{( error } \langle\textit{ string }\rangle \texttt{ ) }\\
& | & \langle\textit{ specific\_success\_response }\rangle\\
\end{array}
\]

\textcolor{red}{TODO}: the example proposed by Manuel

\subsection{Z3}

One of the \acrshort{smt} solvers that implements the SMT-LIB standard is the Z3
theorem prover from Microsoft Research.

\textcolor{red}{TODO}: reference, show its usage with -in and say why we have
chosen it and why we only communicate with it by means of SMT-LIB.

Note that there may exist subtle non-compliances when a solver implements the
SMT-LIB standard. For example, we have found that Z3 does not include the
surrounding double-quotes when it prints back the provided string literal, which
is the specified behavior in the standard.

This may lead to confusion because the \verb|echo| command is the only one whose
response is a string literal and, as this is not the case for Z3, there are 
corner cases in which a command response can be confused with a printed string
intended to delimit command responses, which is one of the proposed usages for
\verb|echo| in \cite{smtLibStandard}:

\textcolor{red}{TODO}: clarify why this affects us

\begin{verbatim}
$ z3 -in <<<`(check-sat) (echo "sat")'
sat
sat
\end{verbatim}
