\chapter{SMT Solver Integration in Elixir}
\label{cap:smtSolverIntegration}

In order to implement our system, we will require to be able to interact with an
\acrshort{smt} solver from Elixir. We have decided to use the Z3 theorem prover,
which supports SMT-LIB, and to communicate with it precisely by using this
standard. This makes possible for our system to allow integrating other
\acrshort{smt} solvers or to provide different communication implementations
without requiring so much effort.

Then, we will introduce a simple formal language whose semantics is defined in
terms of the \gls{smt} problem, and an example of its implementation in Elixir
as a result of the previous integration with the solver.

\section{SMT-LIB interpreter binding}

As we did not find any existing Elixir package that met our requirements
explained above, except for one that seemed to be unmaintained and not ready for
a general-purpose usage, we addressed the implementation of an SMT-LIB
interpreter binding as an opportunity to get started with Elixir in practice,
and also with its macro system. This has given place to a side project which
consists of an Elixir \gls{dsl} to communicate with SMT-LIB interpreters, and
may be eventually provided as a general-purpose library to be available for the
Elixir community.

\subsection{Overview}
\label{integ:dslexample}

By using our solver \gls{dsl}, the SMT-LIB example shown in Section
\ref{prelim:smtlib} can be written in Elixir as follows:

\begin{lstlisting}[language=elixir,numbers=none,frame=none]
import SmtLib

with_local_conn do
  declare_const x: Int,
                y: Int

  assert !((:x + 3 <= :y + 3) ~> (:x <= :y))
  case check_sat do 
    {:ok, :unsat} -> IO.puts("Verified!")
    _             -> IO.puts("Not verified")
  end
end
\end{lstlisting}

which prints \verb|"Verified!"|, proving that 

$$x + 3\leq y + 3 \Rightarrow x \leq y$$

We provide a \verb|with_local_conn/1| macro that creates a default fresh
connection with Z3 through ports, injects it as the first argument to its inner
SMT-LIB command \gls{dsl} macros (e.g. \verb|declare_const/2| and
\verb|assert/2|) and closes it once all of them have been evaluated. It is a
convenient wrapper around another macro, namely \verb|with_conn/2|, which allows
to provide a custom or reused connection and does not close it automatically.

Regarding the contents of a \verb|with_conn/2| block, our \gls{dsl} currently
supports a subset of SMT-LIB commands that is shown in the following section,
but it is almost trivial to add support for new ones, being the biggest
challenge to parse its response if it has a specific one.  The expressions
corresponding to logical formulas include support for variables, uninterpreted
function applications, quantifiers, and built-in operators and logic connectives
such as \verb|+|, \verb|!| and \verb|&&|.

A longer example of using this binding to solve a \gls{csp} is shown in Appendix
\ref{Appendix:hamiltonian}.

\subsection{Implementation}

As explained in Section \ref{prelim:typespecs}, we will use Elixir type
specifications as a guide to explain our implementation in a simplified form.
Also, we will prefer to implement regular Elixir functions and then define 
macros only in the top level module by making use of this regular functions.

First, we have defined \verb|type|s to represent SMT-LIB commands and responses
from the subset that we have shown in Section \ref{prelim:smtlib}:

\begin{lstlisting}[language=elixir,numbers=none,frame=none]
@type command_t ::
  {:assert, term_t}
  | :check_sat
  | {:push, numeral_t}
  | {:pop, numeral_t}
  | {:declare_const, symbol_t, sort_t}
  | {:declare_sort, symbol_t, numeral_t}
  | {:declare_fun, symbol_t, [sort_t], sort_t}

@type general_response_t ::
  :success
  | :unsupported
  | {:error, string_t}
  | {:specific_success_response, specific_success_response_t}
\end{lstlisting}

where other involved types like \verb|numeral_t| and \verb|sort_t| are defined 
similarly, many of them as an alias to built-in Elixir value types.

Then, we have implemented a function that, given a subset of the Elixir AST
(i.e. our \gls{dsl}), transforms it into a list of SMT-LIB commands:

\begin{lstlisting}[language=elixir,numbers=none,frame=none]
@spec ast_to_commands(ast) :: [command_t]
\end{lstlisting}

Its implementation defines cases for each possible term and subterms, like the
following for the \verb|declare-const| command:

\begin{lstlisting}[language=elixir,numbers=none,frame=none]
@spec ast_to_command(SmtLib.ast) :: command_t
def ast_to_command({:declare_const, _, [{v, s}]}) do
  {:declare_const, symbol(v), sort(s)}
end
\end{lstlisting}

In fact, our implementation is a bit more complicated because it allows using 
run-time values bound to Elixir variables in some places, such as constants and
identifiers, thus in our case it translates Elixir \gls{ast} corresponding to
our \gls{dsl} into other Elixir \gls{ast} that evaluates to SMT-LIB commands.

Once we were able to transform the \gls{dsl} into SMT-LIB commands, we required
a function to render each command into a \verb|string| that an SMT-LIB
interpreter understands:

\begin{lstlisting}[language=elixir,numbers=none,frame=none]
@spec command_to_string(command_t) :: String.t
\end{lstlisting}

This function handles compositionally the \verb|command_t| type with cases like
the following:

\begin{lstlisting}[language=elixir,numbers=none,frame=none]
{:declare_const, s1, s2} ->
  "(declare-const #{symbol(s1)} #{sort(s2)})"
\end{lstlisting}

Besides this, in order to understand the solver responses, we have also
implemented a function that parses a received \verb|string|:

\begin{lstlisting}[language=elixir,numbers=none,frame=none]
@spec general_response_from_string(String.t) :: 
  {:ok, general_response_t} 
  | {:error, term}
\end{lstlisting}

This function has been implemented using \verb|NimbleParsec|, an Elixir package
of parser combinators, in order to delegate this task and get the reliability of
a well tested tool \citep{NimbleDocs}. Its top level parser definition is as
follows:

\begin{lstlisting}[language=elixir,numbers=none,frame=none]
defparsec :general_response,
  skip_blanks_and_comments()
  |> choice([
    token(success()) |> eos(),
    token(unsupported()) |> eos(),
    token(error()) |> eos(),
    token(specific_success_response()) |> eos()
  ])
\end{lstlisting}

where, for example, \verb|eos/1| is a combinator that denotes the end of the 
stream of data being parsed, \verb|success/1| parses the success response of
SMT-LIB commands that do not have a specific one, and
\verb|specific_success_response/1| parses specific responses such as the one for
\verb|check_sat|.

Finally, to interact with the solver, we have defined an Elixir low level
protocol to send SMT-LIB commands to a solver and receive SMT-LIB responses in a
synchronous way:

\begin{lstlisting}[language=elixir,numbers=none,frame=none]
defprotocol Connection do
  @spec send_command(t, command_t) :: 
    :ok | {:error, term}
  def send_command(connection, command)

  @spec receive_response(t) 
    :: {:ok, general_response_t} | {:error, term}
  def receive_response(connection)
end
\end{lstlisting}

Elixir protocols are mechanisms to achieve polymorphism and, in order to
implement a connection with an \acrshort{smt} solver, it suffices to define an
Elixir module with a data type that implements the \verb|Connection| protocol
with these two functions. We provide a default implementation of this protocol
to communicate with Z3 through ports and allow the user to configure some of its
parameters like the timeout, but other implementations involving different
solvers and communication mechanisms should also be possible.

All of this allows us to implement and provide the public \gls{api} of the 
package under a top level module that defines the corresponding \gls{dsl} macros
and can be used as in Section \ref{integ:dslexample}.

\section{The L0 language}

This section introduces a formal language that we have named L0. The name stands
for `Level 0', since it is the lowest level language of our verification stack.

L0 is intended to be implemented as Elixir expressions that send SMT-LIB 
commands to an \acrshort{smt} solver. This will allow us to define a
verification \gls{ir} on top of it.

\subsection{Notation}

We assume that $\mathbb{F}$ is the set of many-sorted logic formulae involving
equality, uninterpreted function symbols and arithmetic. We use $\varphi$, 
$\psi$, etc. to denote elements from this set.

Also, we assume a set $\Sigma^{0}$ of uninterpreted function symbols and a set
$\mathbb{T}$ of terms in many-sorted logic, generated by the following grammar:

\[
\mathbb{T} \ni t ::= n \mid x \mid f(t_1, \ldots, t_m)
\]

where $n$ is a number, $x$ is a variable, and $f \in \Sigma^{0}$ is a function 
symbol of arity $m$.

\subsection{Syntax}

The syntax of L0 expressions is given by the following grammar:

\[
\begin{array}{rcll}
\Exp{0} \ni \epsilon & ::= & \skipE & \{ \textrm{do nothing} \}\\
& | & \failE & \{ \textrm{fail signal} \}\\
& | & \epsilon_1;\epsilon_2 & \{ \textrm{sequential evaluation} \}\\
& | & \localE~\epsilon & \{ \textrm{local scoped proof state} \}\\
& | & \addE~\varphi &  \{ \textrm{add a logic formula $\varphi \in \mathbb{F}$ to the state} \}\\
& | & \declareE{x} &  \{ \textrm{declare a variable of type $\Term$} \}\\
& | & \whenUnsatE{\epsilon_1}{\epsilon_2}{\epsilon_3} &  \{ \textrm{unsatisfiability conditional} \}\\
\end{array}
\]

where $\Term$ is a sort that must be previously defined in the \acrshort{smt}
solver.

If $I = [i_1, \ldots, i_n]$ is a sequence of elements, we use the notation
$\overline{\epsilon_i}^{i \in I}$ to denote the sequential composition 
$\epsilon_{i_1};\dots;\epsilon_{i_n}$.

As we will reflect in the language semantics, the $\failE$ expression will allow
us to trigger a validation failure, the $\declareE{x}$ and $\addE~\varphi$ ones
will allow us to respectively declare a new variable and to add a formula to the
state, the $\localE$ one to evaluate an expression without incorporating its
changes in the final state, and $\whenUnsatW$ will be a control flow construct
that depends on the unsatisfiability of a given expression.  Sequential
evaluation will combine expressions to be evaluated one after the other, and
$\skipE$ will be useful for example if some $\whenUnsatW$ branch requires doing
nothing. 

\subsection{Semantics}

Let $V$ be a set of variable names, $\mathbb{F}(V)$ the subset of $\mathbb{F}$ 
with free variables in $V$, and a predicate \textit{unsat} which, given a set of
formulas $\Phi$ from $\mathbb{F}$, determines whether they are unsatisfiable or
not.  We define the big step operational semantics of L0 expressions as the
smallest relation $\BS{\epsilon}{X}{\Phi}{(X', \Phi')}$ between $\Exp{0} \times
\mathcal{P}(V) \times \mathcal{P}({\mathbb{F}(V)})$ and $\mathcal{P}(V) \times
\mathcal{P}({\mathbb{F}(V)})$ that satisfies the rules from Figure
\ref{fig:l0bs}.

A pair $(X, \Phi)$ denotes the state of the \acrshort{smt} solver, where $X$ is 
the set of variable names defined at the moment, and $\Phi$ the set of formulas
that have been added also at that moment. A judgement
$\BS{\epsilon}{X}{\Phi}{(X', \Phi')}$ means that the expression $\epsilon$
transforms the solver state $(X, \Phi)$ into the state $(X', \Phi')$. 

\begin{figure}
\begin{prooftree}
  \AxiomC{ }
  \UnaryInfC{$\BS{\skipE}{X}{\Phi}{(X, \Phi)}$}
\end{prooftree}

\begin{center}
\begin{minipage}[t]{0.45\textwidth}
  \begin{prooftree}
    \AxiomC{$\varphi \in \mathbb{F}(X)$}
    \UnaryInfC{$\BS{\addE~\varphi}{X}{\Phi}{(X, \Phi \cup \{\varphi\})}$}
  \end{prooftree}
\end{minipage}
\begin{minipage}[t]{0.45\textwidth}
  \begin{prooftree}
    \AxiomC{$x \notin X$}
    \UnaryInfC{$\BS{\declareE{x}}{X}{\Phi}{(X \cup \{x\}, \Phi)}$}
  \end{prooftree}
\end{minipage}
\end{center}

\medskip

\begin{prooftree}
  \AxiomC{$\BS{\epsilon_1}{X}{\Phi}{(X', \Phi')}$}
  \AxiomC{$\BS{\epsilon_2}{X'}{\Phi'}{(X'', \Phi'')}$}
  \BinaryInfC{$\BS{\epsilon_1;\epsilon_2}{X}{\Phi}{(X'', \Phi'')}$}
\end{prooftree}

\medskip

\begin{prooftree}
  \AxiomC{$\BS{\epsilon}{X}{\Phi}{(X', \Phi')}$}
  \UnaryInfC{$\BS{\localE~\epsilon}{X}{\Phi}{(X, \Phi)}$}
\end{prooftree}

\medskip

\begin{prooftree}
  \AxiomC{$\BS{\epsilon_1}{X}{\Phi}{(X', \Phi')}$}
  \AxiomC{$\unsat{\Phi'}$}
  \AxiomC{$\BS{\epsilon_2}{X}{\Phi}{(X'', \Phi'')}$}
  \TrinaryInfC{$\BS{\whenUnsatE{\epsilon_1}{\epsilon_2}{\epsilon_3}}{X}{\Phi}{(X'', \Phi'')}$}
\end{prooftree}

\medskip

\begin{prooftree}
  \AxiomC{$\BS{\epsilon_1}{X}{\Phi}{(X', \Phi')}$}
  \AxiomC{$\neg\unsat{\Phi'}$}
  \AxiomC{$\BS{\epsilon_3}{X}{\Phi}{(X'', \Phi'')}$}
  \TrinaryInfC{$\BS{\whenUnsatE{\epsilon_1}{\epsilon_2}{\epsilon_3}}{X}{\Phi}{(X'', \Phi'')}$}
\end{prooftree}
\caption{Big step operational semantics of the L0 language}
\label{fig:l0bs}
\end{figure}

The absence of rules for the $\failE$ expression is intentional, because we want
any reachable $\failE$ to prevent evaluation of the expression in which it is
contained.  We have also required this to happen if the same variable is
declared twice or if a formula with undeclared variables is being added to the 
solver's state.

Note also that a $\localE$ expression discards the variable definitions and
formulas added by its expression, restoring the initial state, and the same
happens for the expression under the unsatisfiability test in a $\whenUnsatW$
expression.

\subsection{Implementation}
\label{integ:l0implementation}

It is difficult to justify the compliance of an implementation of L0 with its
formal semantics due to the undecidability of the \gls{smt} problem in the 
general case. In practice, this task will be delegated to an \acrshort{smt} 
solver as if it were a black box that can determine whether a set of first-order
formulas is unsatisfiable.

We can implement a simple Elixir \gls{dsl} for the L0 language in terms of our
SMT-LIB binding for Elixir. The \verb|fail| expression raises an exception:

\begin{lstlisting}[language=elixir,numbers=none,frame=none]
defmacro eval(_, {:fail, _, _}) do
  quote do
    raise "Verification failed"
  end
end
\end{lstlisting}

The \verb|local| expression surrounds the evaluation in between \verb|pop| and
\verb|push| SMT-LIB commands:

\begin{lstlisting}[language=elixir,numbers=none,frame=none]
defmacro eval(conn, {:local, _, [e]}) do
  quote do
    conn = unquote(conn)
    :ok = push conn
    eval conn, unquote(e)
    :ok = pop conn
  end
end
\end{lstlisting}

The \verb|add| expression corresponds to an \verb|assert| in SMT-LIB:

\begin{lstlisting}[language=elixir,numbers=none,frame=none]
defmacro eval(conn, {:add, _, [f]}) do
  quote do
    conn = unquote(conn)
    :ok = assert conn, unquote(f)
  end
end
\end{lstlisting}

Similarly, the \verb|declare_const| expression corresponds to a
\verb|declare-const| in SMT-LIB:

\begin{lstlisting}[language=elixir,numbers=none,frame=none]
defmacro eval(conn, {:declare_const, _, [x]}) do
  quote do
    conn = unquote(conn)
    :ok = declare_const conn, [{unquote(x), Term}]
  end
end
\end{lstlisting}

The \verb|when-unsat| expression implementation is slightly longer:

\begin{lstlisting}[language=elixir,numbers=none,frame=none]
defmacro eval(
  conn, 
  {:when_unsat, _, [e1, [do: e2, else: e3]]}
) do
  quote do
    conn = unquote(conn)
    :ok = push conn
    eval conn, unquote(e1)
    {:ok, result} = check_sat conn
    :ok = pop conn

    case result do
      :unsat -> eval conn, unquote(e2)
      _ -> eval conn, unquote(e3)
    end
  end
end
\end{lstlisting}

Finally, instead of implementing a \verb|seq| expression, we can reuse Elixir
blocks by handling several cases, and allowing to provide them with \verb|do|
syntax (i.e. to write an Elixir block argument within a trailing \verb|do| and
\verb|end| delimiters in a macro invocation):

\begin{lstlisting}[language=elixir,numbers=none,frame=none]
defmacro eval(
  conn, 
  do: {:__block__, [], []}
) when is_list(es) do
  nil
end

defmacro eval(
  conn, 
  do: {:__block__, [], [e | es]}
) when is_list(es) do
  quote do
    conn = unquote(conn)
    eval conn, unquote(e)
    eval conn, unquote({:__block__, [], [es]})
  end
end

defmacro eval(conn, do: e) do
  quote do
    conn = unquote(conn)
    eval conn, unquote(e)
  end
end
\end{lstlisting}

We can also include a general case that raises an exception if the provided
Elixir \gls{ast} does not correspond to our language:

\begin{lstlisting}[language=elixir,numbers=none,frame=none]
defmacro eval(_, other) do
  raise "Unknown expression #{Macro.to_string(other)}"
end
\end{lstlisting}

Assuming the defined macros to be in scope, and a \verb|conn| variable that
represents a fresh connection with an \acrshort{smt} solver which has the
\verb|Term| sort already defined, the following is a simple example of the usage
of the \verb|eval/1| macro:

\begin{lstlisting}[language=elixir,numbers=none,frame=none]
eval conn do
  declare_const :x

  # Replacing `!=' by `==' leads to a verification
  # exception, since `fail' is executed
  when_unsat add :x != :x do
    skip
  else
    fail
  end
end
\end{lstlisting}
